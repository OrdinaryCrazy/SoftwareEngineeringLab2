\chapter{软件质量特性}
%======================================================================================
% 详细说明项目任何其他的质量特性。该特性对客户和开发者都非常重要。
% 考虑的方面包括:适应性,可用性,正确性,灵活性,交互工作能力,可维护性,可移植性,可靠性,
% 可重用性,鲁棒性,可测试性和可用性等。定量的详细描述这些特性,尽可能的可验证。
% 对不同属性之间的重要性加以阐述,如:易用性比易学性更重要。
% 每一个属性单独使用一个小节描述,可根据需要进行增减,如增加可维护性小节等。
%======================================================================================
\section{正确性}
    用户操作,信息发送,文件管理等操作要求准确无误,不产生错误信息,不丢失信息。
\section{可靠性}
    不存在闪退,死锁,操作无效等情况。
\section{效率}
    各种操作,特别是消息传递在尽可能短的时间内完成。
\section{完整性}
    出现意外情况,比如网络环境突变,设备掉电等情况是保证数据完整一致。
\section{易使用性}
    符合所在平台操作风格,各种操作简便人性化,教程简单易懂,绝大部分功能可以轻松使用。
\section{可维护性}
    数据库和代码可以方便地维护和修改。
\section{可测试性}
    提供高效简洁的测试接口和测试方法,减少测试困难程度和工作量。
\section{复用性}
    尽量实现代码的可重用性,设计时尽量模块化。做到高内聚,低耦合。
\section{安全保密性}
    要求提供身份验证,只有通过身份验证才能使用软件;能够抵抗恶意攻击,保护隐私数据和受保护内容不被盗取;系统不会被恶意攻击至瘫。
\section{可理解性}
    提供完全图形化界面,各种提示、操作信息直观易懂,符合常识,尽可能方便用户操作。
\section{互联性}
    要求网络条件正常,不低于10KB/s时,消息传送正常,各应用没有严重的延迟。


