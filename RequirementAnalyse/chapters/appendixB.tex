\chapter{需求建模 }
%======================================================================================
    \section{数据流图}

        \subsection{顶层数据流图}
        %======================================================================================
        % 在这里画出顶层数据流图
        %======================================================================================

        \subsection{层数据流图}
        %======================================================================================
        % 在这里画出0层数据流图
        %======================================================================================

%======================================================================================
    \section{数据字典}

        \subsection{数据流说明}
            
            %======================================================================================
            % 与数据流图中的名称一致,采用数据描述符号说明数据流的内容
            %===========================================================
            % 个人用户
            %======================================================================================
            \subsubsection{音视频信息(out)}
                内容说明:音视频信息(out) = [音频聊天请求(out) | 视频聊天请求(out) | 音频聊天数据流 | 视频聊天数据流]
            \subsubsection{音视频信息(in)}
                音视频信息(in) = [音频聊天请求(in) | 视频聊天请求(in) | 音频聊天数据流 | 视频聊天数据流]
            \subsubsection{好友请求(out)}
                好友请求(out) = [申请添加联系人信息 | 申请加入群组信息 | 同意添加申请人信息 | 同意加入群组信息]
            \subsubsection{好友请求(in)}
                好友请求(in) = [被申请加为联系人信息 | 被邀请加入群组信息]
            \subsubsection{聊天消息(out)}
                聊天信息(out) = 己方账号 + \{字符,图片\} + 发出时间
            \subsubsection{聊天消息(in)}
                聊天信息(in) = 对方账号 + \{字符,图片\} + 接收时间
            \subsubsection{消息查询请求}
                消息查询请求 = 己方账号 + 被查询方账号 + 查询内容
            \subsubsection{文件(out)}
                文件(out) = 己方账号 + 文件 + 发出时间
            \subsubsection{文件(in)}
                文件(out) = 对方账号 + 文件 + 接收时间
            \subsubsection{协同文档内容(out)}
                协同文档内容(out) = 己方账号 + 修改内容 + 修改时间
                修改内容 = \{字符,图片\}
            \subsubsection{协同文档内容(in)}
                协同文档内容(out) = 当前文档
                当前文档 = \{字符,图片\}
            \subsubsection{个人信息}
                个人信息 = 己方账号 + 己方档案
            \subsubsection{个人日程信息}
                个人日程信息 = 己方账号 + 日程信息变动
                日程信息变动 = [增加的日程 | 变更的日程 | 删除的日程]
            \subsubsection{日历表}
                日历表 = \{时间 + 事件\}
            \subsubsection{好友推荐信息}
                好友推荐信息 = 对方账号 + 对方档案 + 联系数量
            \subsubsection{联系人列表}
                联系人列表 = \{联系人档案\}
            %=======================================================
            % 团队用户
            %===========================================================
            \subsubsection{任务/活动}
                任务/活动 = \{发布者信息 + [任务 | 活动] + 发布时间\}
            \subsubsection{群组信息}
                团队信息 = 成员名单变动
                成员名单变动 = [添加的成员名单 | 删除的成员名单]
            \subsubsection{权限信息}
                权限信息 = \{成员账号 + 成员权限\}
            \subsubsection{完成情况}
                完成情况 = 当前时间 + \{成员账号 + 成员已完成比例\}
            \subsubsection{群组人员反馈信息}
                成员反馈信息 = \{反馈者账号 + 反馈内容 + 反馈时间\}
            \subsubsection{团队用户信息}
                团队用户信息 = \{团队成员 + 权限信息 + 任务活动历史\}
                权限信息 = \{成员账号 + 成员权限\}
                任务活动历史 = \{任务活动名称 + 任务活动内容 + 发布时间 + 完成时间\}
            %=======================================================
            % 通讯公司
            %===========================================================
            \subsubsection{用户信息}
                用户信息 = \{用户档案\}
            \subsubsection{好友推荐名单}
                好友推荐名单 = \{推荐好友档案\}
            

        \subsection{数据存储说明}
            \subsubsection{本地文件存储}
            \begin{itemize}
                \item 说明:存储用户之间传送的文件,位于本地的存储空间中
                \item 编号:D1
                \item 输入数据流:[上传的群文件,上传的日历文件, 上传的文件]
                \item 输出数据流:[下载的群文件,下载的日历文件, 下载的文件]
                \item 排列方式:群文件按照群分组,按照日期排列,索引为群账号+文件名\\
                               个人文件按照发送用户分组,按照日期排列,索引为用户账号+文件名\\
                               日历文件按照用户分组,索引为用户账号+日历标记\\
            \end{itemize}
            \subsubsection{云文件存储}
            \begin{itemize}
                \item 说明:存储用户之间传送的文件,位于云端的存储空间中
                \item 编号:D2
                \item 输入数据流:云文件上传请求
                \item 输出数据流:云文件下载请求
                \item 排列方式:群文件按照群分组,按照日期排列,索引为群账号+文件名\\
                               个人文件按照发送用户分组,按照日期排列,索引为用户账号+文件名\\
                               日历文件按照用户分组,索引为用户账号+日历标记\\
                               存储单元为发送者+(接收者)+文件名+文件\\
            \end{itemize}
            \subsubsection{消息记录存储}
            \begin{itemize}
                \item 说明:存储用户的消息记录
                \item 编号:D3
                \item 输入数据流:[聊天记录,查询请求]
                \item 输出数据流:记录消息
                \item 排列方式:按照发送者分组,按照日期排列,索引为发送者+发送时间+接收者\\
                         存储单元为发送者+发送时间+接收者+聊天记录\\
            \end{itemize}
            \subsubsection{用户信息存储}
            \begin{itemize}
                \item 说明:存储用户的档案信息,例如账号,姓名,联系方式等
                \item 编号:D4
                \item 输入数据流:[个人用户信息,团队用户信息]
                \item 输出数据流:[下载的用户信息]
                \item 排列方式:个人用户信息按照用户账号排列,索引为用户账号\\
                               团队用户信息按照团队账号排列,索引为团队账号\\
                               存储单元为账号+[个人标识,团队标识]+联系人列表+头像+邮箱+手机号+个人简介\\
            \end{itemize}
            \subsubsection{协作文档存储}
            \begin{itemize}
                \item 说明:存储过期时间之前的协作文档
                \item 编号:D5
                \item 输入数据流:[初始文档, 当前文档]
                \item 输出数据流:下载的文档
                \item 排列方式:按照发布者账号分组,组内按照最近更改时间排序,索引为发布者账号+文档名\\
                         存储单元为发布者账号+文档名+发布时间+最近更改时间+成员信息+文档内容\\
            \end{itemize}
            %======================================================================================
            % 与数据流图中的名称一致,采用数据描述符号说明数据流的内容,另外还需描述数据排列方式
            %======================================================================================
        \subsection{加工说明}
            \subsubsection{文件管理}
            \begin{itemize}
                \item 说明:响应用户的传输文件需求,并将文件组织好放置于存储中
                \item 输入数据流:文件(in), 下载的文件
                \item 输出数据流:文件(out), 上传的文件
                \item 激发条件:有文件传输请求;有用户的下载请求
                \item 处理过程:\\
                      IF(有文件传输请求):\\
                        检验发送方合法性;\\
                        检验接受方合法性;\\
                        接受文件;\\
                        将文件包装为 发送方 文件 接受方 发送时间的格式;\\
                      IF(本地存储已满) \\
                        将文件上传到云端存储中\\
                      ELSE \\
                        将文件上传到本地存储中;\\
                        发送文件;\\
                
                      IF(有用户下载请求):\\
                        检验用户合法性;\\
                        计算文件存储地址;\\
                        IF(文件不存在) 向用户返回错误信息;\\
                        ELSE IF(地址在本地)访问本地文件存储\\
                        ELSE 向云文件管理发送云文件下载请求;\\
                      接受文件;\\
                      发送文件\\
            \end{itemize}    
            \subsubsection{云文件管理}
            \begin{itemize}
                \item 说明:管理云端文件的上传和下载
                \item 输入数据流:[云文件上传请求,云文件下载请求,下载的云文件]
                \item 输出数据流:云文件,上传的云文件
                \item 激发条件:收到云文件上传请求或云文件下载请求
                \item 处理过程:  \\
                IF(云文件上传请求):\\
                    检验用户合法性;\\
                    分配地址;\\
                    上传云文件\\
                IF(云文件下载请求):\\
                    检验用户合法性;\\
                    计算物理地址;\\
                    取文件;\\
                    返回云文件\\
            \end{itemize}
            \subsubsection{日历管理}
            \begin{itemize}
                \item 说明:个人用户制订个人日历,团队用户可以在该成员的日历上添加任务与活动信息
                \item 输入数据流:[个人日程信息,团队日程信息,下载的日程信息]
                \item 输出数据流:[日历表,上传的日程信息,冲突信息反馈]
                \item 激发条件:有个人日程信息或团队日程信息输入
                \item 处理过程:   \\
                      IF(输入个人日程信息):\\
                        检查与原日历表是否冲突;\\
                      IF(冲突) 向个人返回冲突信息;\\
                      ELSE:\\
                        向日日历中加入该日程;\\
                        上传新的日历文件;\\
                        返回新的日历文件\\
                     ELSE(输入团队日程信息)\\
                    FOR(成员 in 团队)\\
                        检查与该成员原日历表是否冲突;\\
                        IF(冲突) 向团队返回冲突信息;\\
                    
                    FOR(成员 in 团队)\\
                        向该成员日历中加入该日程;\\
                        上传新的日历文件;\\
                        向该成员发送新的日历文件\\
            \end{itemize}
            \subsubsection{团队管理}
            \begin{itemize}
                \item 说明:管理团队的成员,发布与管理团队的任务活动
                \item 输入数据流:[成员信息,权限信息,任务活动,冲突信息反馈,下载的群文件]
                \item 输出数据流:[团队用户信息,成员反馈信息,初始文档,团队日程信息,上传的群文件,]
                \item 激发条件:有成员信息,权限信息,任务活动传入
                \item 处理过程:   \\
                IF(成员信息):\\
                    IF(添加成员):\\
                        向团队用户信息中添加成员\\
                        输出团队用户信息\\
                    ELSE IF(删除成员):\\
                        从团队用户信息中删除成员\\
                        输出团队用户信息\\
                    ELSE IF(修改成员):\\
                        在团队用户信息中修改成员\\
                        输出团队用户信息\\
                    ELSE IF(查询成员):\\
                        从存储中下载成员信息\\
                        输出查询成员信息\\
                IF(权限信息):\\
                    修改对应成员的权限信息\\
                    更新团队信息\\
                IF(任务活动):\\
                    发出团队日程信息;\\
                    IF(是文档任务)\\
                        发布初始文档\\
                    更新团队信息\\
                IF(冲突信息反馈):\\
                    重新指定对应任务\\
                    发布新的团队日程信息\\
                    更新团队信息    \\
                IF(群文件):\\
                    上传群文件\\
                    
            \end{itemize}
            \subsubsection{通讯管理}
            \begin{itemize}
                \item 说明:基本功能。用户之间发送信息进行通讯
                \item 输入数据流:[聊天消息(out),音视频信息(out),消息查询请求,记录消息]
                \item 输出数据流:[聊天消息(in),音视频信息(in),聊天记录,查询请求]
                \item 激发条件:有消息输入;有用户上线
                \item 处理过程: \\
                    IF(传入聊天消息):\\
                        IF(对方用户在线):\\
                            立刻转发给对方用户\\
                        ELSE:\\
                            缓存信息 对方用户上线时转发给对方\\
                        将聊天记录上传到存储中\\
                    ELSE IF(传入音视频信息):\\
                        立即发送给对方\\
                        IF(未收到对方反馈):\\
                            中断音视频连接\\
                    ELSE IF(消息查询请求):\\
                        从存储中下载记录消息\\
                        将记录消息返回\\
                    ELSE IF(用户上线)\\
                        查询缓存,发送离线信息\\
            \end{itemize}
            \subsubsection{用户信息管理}
            \begin{itemize}
                \item 说明:实现用户信息的绑定,好友推荐的信息处理,好友请求,联系人列表管理等和用户信息有关的功能
                \item 输入数据流:[好友请求(out),好友推荐名单,个人信息,下载的用户信息]
                \item 输出数据流:[好友请求(in),好友推荐信息,个人用户信息]
                \item 激发条件:有信息输入;有用户上线
                \item 处理过程: \\
                IF(好友请求)\\
                    IF(对方用户在线):\\
                        立刻转发给对方用户\\
                    ELSE:\\
                        缓存信息 对方用户上线时转发给对方\\
                ELSE IF(个人信息)\\
                    将个人信息上传到用户信息存储中,绑定到对应的用户\\
                ELSE IF(好友推荐名单)\\
                    下载名单中用户的用户信息\\
                    将推荐用户及其部分信息转发给对应用户\\
                ELSE IF(有用户上线)\\
                    将缓存的离线信息发送给对应用户\\
             \end{itemize}
            \subsubsection{在线文档协作平台}
            \begin{itemize}
                \item 说明:可以多人同时编辑文档
                \item 输入数据流:[协作文档内容(out),下载的文档]
                \item 输出数据流:[协作文档内容(in),当前文档,完成情况]
                \item 激发条件:有文档/文档的增删改查传入
                \item 处理过程: \\
                IF(布置新的文档)\\
                    更新完成情况\\
                ELSE IF(文档增加内容)\\
                    增加文档\\
                    返回最新的协作文档内容\\
                    上传当前文档\\
                    更新完成情况\\
                ELSE IF(文档删除内容)\\
                    删除文档\\
                    返回最新的协作文档内容\\
                    上传当前文档\\
                    更新完成情况\\
                ELSE IF(文档修改内容)\\
                    修改文档\\
                    返回最新的协作文档内容\\
                    上传当前文档\\
                    更新完成情况\\
            \end{itemize}
            \subsubsection{错误信息处理单元}
            \begin{itemize}
                \item 说明:系统中的每个元件返回的错误信息,例如存储空间不足,系统版本不兼容等
                \item 输入数据流:所有元件的错误信息(因为每个元件都有,为了图的简洁,在图中未标出)
                \item 输出数据流:系统错误反馈
                \item 激发条件:有错误信息到来
                \item 处理过程:\\
                    IF(错误信息):\\
                        记录错误来源与错误类型\\
                        向公司反馈错误\\
                        采用应急预案\\
            \end{itemize}

            %======================================================================================
            % 采用自然语言,判断表/判断树,伪码的形式描述对数据流进行处理的过程
            %======================================================================================
%======================================================================================