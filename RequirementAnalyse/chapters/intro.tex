\chapter{简介}
\section{目的}
% This section should state the purpose of the document. It could also specify the intended audience. Identify the product whose software requirements are specified in this document.

% 这部分要描述文档的目的。应该指明读者。说明本需求文档描述了哪个产品的软件需求。

现代工作团队,包括高校学生团体,课堂组织,班级管理,科研团队,企业部门,开发团队,工作小组等对于即时通讯系统
有着专业的、高质量的应用需求,提供高效、专业,功能强大、扩展性好的即时通讯系统对于提高
工作团队工作效率和管理水平有着重要意义。本产品将降低信息管理和通讯代价,为不同情境的通信需求
提供提供解决方案,包括一对一即时通讯(私聊)功能、情境群聊功能、个人日程管理、活动/任务发布,音视频通话/会议、
广播/公告板功能,个性化好友推荐,在线文档写作平台等工作团队需求度较高的需求解决方案。

\section{范围}
% This section should address areas which this document includes and that are specifically excludes. 

% 本节应描述文档所包括和不包括的内容。
    本文档包括对于用户需求的分析和产品功能的介绍,描述各项功能的具体需求,约束和限制,流程,依赖关系,
    为即时通讯系统的设计、实现、测试以及验收提供重要依据,也为评价系统功能和性能
    提供标准。本文档可供用户、项目管理人员、系统分析人员、程序设计人员以及
    系统测试人员阅读和参考。
