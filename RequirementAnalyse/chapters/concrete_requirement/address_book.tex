\subsection{R.INTF.CALC.005: 通讯录功能}
在日常使用中,我们一般不会和陌生人通信,而是和好友进行联系,或是在加入的群聊中发言。于是,我们使用通讯录管理所有的好友和群聊。
\subsubsection{介绍}
对用户而言,该功能的需求为:
\begin{itemize}
  \item 用户可以添加其他人为好友,可以通过账号查询、二维码等多种方式添加好友,向其发送好友请求。
  \item 用户可以加入群聊,可以通过账号查询、二维码等多种方式加入群聊,向其发送入群申请。
  \item 用户可以接受或拒绝其他用户的好友请求和入群邀请。
  \item 用户可以在通讯录中查询好友或群聊,进行通讯。
  \item 用户可以单方面删除好友。
  \item 用户可以退出群聊。
  \item 用户可以为好友进行分组。
  \item 用户可以将其他人加入或移出黑名单,用户不会接收来自黑名单的任何信息。
  \item 用户可以为好友增加备注信息。
\end{itemize}
\subsubsection{输入}
用户对好友的添加、删除和查询,用户设置的备注信息。
\subsubsection{处理}
\begin{itemize}
  \item \textbf{添加好友(群聊):} 可以根据提供的信息定位到具体的用户(群聊),向他发送请求。在他接受后,把他的信息加入通讯录。
  \item \textbf{删除好友:} 从用户的通讯录中删除该好友,同时从他的通讯录中删除用户信息。
  \item \textbf{退出群聊:} 从用户的通讯录中删除该群聊,同时从它的群成员中删除用户信息。
  \item \textbf{查询好友(群聊):} 从通讯录中找到对应的好友(群聊),可以采用遍历、二分查找、哈希等方法。
  \item \textbf{备注好友:} 为好友增加备注。
  \item \textbf{好友分组:} 在通讯录中对好友分组管理。
  \item \textbf{黑名单:} 屏蔽来自黑名单的所有信息。
\end{itemize}
\subsubsection{输出}
展示用户在上述操作后的新通讯录。
