\subsection{R.INTF.CALC.003: 活动/任务发布与管理功能}
在一个组织中,有许多的集体活动或任务,它们一般由多人参与,对每个人而言,时间、地点等信息都相同。于是,我们提供了该功能,使得组织的管理者可以发布和管理任务,并自动通知所有参与者,直接插入他们的日历中。
\subsubsection{介绍}
对用户而言,该功能的需求为:
\begin{itemize}
  \item 组织的管理者可以发布任务,设置任务的参与者、截止时间、地点、具体内容等信息。
  \item 组织的管理者可以修改或删除已经发布的任务。
  \item 任务的参与者会自动接收到任务信息,任务的截止时间等重要时间节点会被自动插入到参与者的日历中。如果任务被删改,每一次改动的具体信息也都会被自动接收,且日历中也会做出相应的改动。
  \item 应当有一个独立的入口可以查看所有的任务,并提供不同的筛选和排序方式。
\end{itemize}
\subsubsection{输入}
用户对任务的增、删、改、查。
\subsubsection{处理}
\begin{itemize}
  \item 使用数据库管理所有的任务,实现增、删、改、查功能,注意针对不同的用户身份提供不同的权限。
  \item 在任务发生修改时,将修改的内容告知所有参与者。
\end{itemize}
\subsubsection{输出}
\begin{itemize}
  \item 在任务管理界面展示任务增、删、改、查后的结果。
  \item 一旦任务发生修改,向用户展示任务修改的内容。可以考虑调用邮件、短信等接口,使用电子邮件或短信来告知用户。
  \item 调用日历接口,自动插入、删除、修改日历中的活动项。
\end{itemize}
