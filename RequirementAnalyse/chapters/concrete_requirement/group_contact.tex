\subsection{R.INTF.CALC.002: 多情境群聊功能}
在日常使用中,用户经常会有群聊的需求。群聊并不是聊天人数的简单增加,而是聊天场景的扩展——从一对一的聊天扩展为多人的聊天场景。我们对多种群聊场景提供支持,如课程、工作小组等。
\subsubsection{介绍}
对用户而言,该功能的基本需求为:
\begin{itemize}
  \item 群主可以创建或解散群聊,并为群聊选择场景。
  \item 群主可以任命群成员为管理员,或解除管理员的权限。
  \item 群主和管理员可以批准或拒绝入群申请。
  \item 群主和管理员可以邀请其他人加入群聊。
  \item 任意群成员在群聊中的所有发言都会被所有群成员看到。
\end{itemize}

在每一个群聊场景中,用户对此有一些额外的需求:

\textbf{课程:}
\begin{itemize}
  \item 课程作业需求。
  \item 课堂签到需求。
  \item 考试成绩、平时作业成绩查询需求。
\end{itemize}

\textbf{工作小组:}
\begin{itemize}
  \item 工作进度管理需求。
  \item 工作汇报协同制作需求。
\end{itemize}
\subsubsection{输入}
用户对群聊的管理和在群聊中的发言,以及对额外功能的使用。
\subsubsection{处理}
\begin{itemize}
  \item \textbf{群管理:} 对群成员列表进行增、删、改操作。
  \item \textbf{群聊天:} 向所有群成员发送该发言信息,如果某成员不在线,则在服务器中缓存该信息,待其上线后发送。
  \item \textbf{群聊场景的额外需求:}
  \begin{itemize}
    \item 对课程作业需求,可以调用任务管理接口实现。
    \item 对课堂签到需求,可以通过在指定时间进行GPS定位,判断用户是否位于指定地点实现。
    \item 对成绩查询需求,可以通过只允许特定的用户查看成绩文件的特定部分实现(每一个人只能看到自己的成绩条目)。
    \item 对工作进度管理需求,可以调用任务管理接口实现。
    \item 对工作汇报协同制作需求,可以调用在线文档写作平台实现。
  \end{itemize}
\end{itemize}
\subsubsection{输出}
\begin{itemize}
  \item \textbf{群管理:} 展示最新的群管理信息。
  \item \textbf{群聊天:} 展示群聊中所有成员的发言信息。
  \item \textbf{群聊场景的额外需求:} 展示使用后的信息,如作业的内容、上交时间、评分标准等。
\end{itemize}