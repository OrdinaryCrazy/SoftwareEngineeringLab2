\subsection{R.INTF.CALC.004: 音视频通话(会议)功能}
在我们的日常使用中,仅仅通过文字进行聊天并不能够满足我们的需求,因此我们提供了音视频通话(会议)功能。
\subsubsection{介绍}
对用户而言,该功能的需求为:
\begin{itemize}
  \item 允许用户向其他用户发起建立连接的请求,以开始音视频通话
  \item 用户可以接受或拒绝其他用户发起的连接请求
  \item 双方的音视频信息应当实时传输
  \item 允许用户随时单方面停止连接
\end{itemize}
\subsubsection{输入}
用户通过点击按钮发起连接请求或停止连接,接受或拒绝对方的连接请求。
\subsubsection{处理}
\begin{itemize}
  \item 权限检查:检查是否拥有所有必需的权限,如视频连接所需的摄像头权限,音频连接所需的麦克风和扬声器权限。如果缺少权限,就向用户请求对应的权限。
  \item 向对方发起建立连接的请求,等待对方接受。
  \item 调用摄像头或麦克风对用户的图像或声音进行实时录制,并发送给连接的另一方。
  \item 为了减少流量,传输中的音视频内容应当被压缩,可以采用现有的成熟的压缩算法。
  \item 对接收到的数据进行缓存,再播放,保证播放不会乱序。
  \item 如果发生用户点击按钮停止连接的事件,就断开连接,停止服务。
\end{itemize}
\subsubsection{输出}
\begin{itemize}
  \item 如果没有得到必需的权限,则停止服务,退出该模块。
  \item 如果请求被对方拒绝,则停止服务,退出该模块。
  \item 使用屏幕播放视频或扬声器播放音频,直到用户停止连接。
\end{itemize}
