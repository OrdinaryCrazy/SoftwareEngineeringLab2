\chapter{总体设计约束}
%======================================================================================
% 描述可能限制开发人员选择的事项。
%======================================================================================
\section{标准符合性}
%======================================================================================
% 本节详细说明需求所采用的标准或规范的来源。
% 如果项目采用了国际标准,应该说明国际标准及项目与标准的偏离情况。
%======================================================================================
开源许可证:GNU General Public License v3.0
无其他标准或规范来源
\section{硬件约束}
%======================================================================================
% 本节包括软件在不同的硬件平台运行的需求,如时间相关的约束,内存方面的约束等。
%======================================================================================
\subsection{移动端}
	\begin{enumerate}
		\item 处理器要求:MSM800系列,Exynos5433以上,HelioX10以上,麒麟系列,A8以上
		\item 内存要求:512MB以上
	\end{enumerate}
\subsection{PC端}
	\begin{enumerate}
		\item 处理器要求:Intel® Core™ i5以上
		\item 内存要求:512MB以上 
	\end{enumerate}
\section{技术限制}
%======================================================================================
% 本节包括对使用特定技术的限制,包括接口,数据库,并行操作,通讯协议,设计约定,编程规范等。
%======================================================================================
\noindent
数据库:Oracle 18.3 \\
编程规范:\\
    1. 首字母小写的驼峰命名法\\
    2,函数大括号换行\\