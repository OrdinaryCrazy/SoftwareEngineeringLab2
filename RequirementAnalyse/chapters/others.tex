\chapter{\color{red}其他需求}
%======================================================================================
% 使用适当的章节,详细说明任何其他客户需求,包括数据库,编码需求,错误处理,测试需求等。
%======================================================================================
\section{编码需求与代码可维护性}
%======================================================================================
    统一Unicode编码;符合JAVA程序开发规范,清晰简明易维护,注释量不少于总代码量40\%;安装程序
    自动检测平台依赖,一键安装方便快捷,一键卸载不留残余。
%======================================================================================
\section{错误处理}
%======================================================================================
\noindent
    客户端: \\
        1. 产生的任何错误不能损害用户数据或损害平台上的其他数据\\
        2. 在设备掉电、系统崩溃的情况下保护好用户数据\\
        3. 及时向服务器发送错误信息\\
    服务器端:\\
        1. 使用安全稳定可靠的第三方服务器\\
        2. 及时处理客户端发送的错误信息维护代码
%======================================================================================
\section{增量更新能力}
%======================================================================================
    客户端软件升级/更新支持增量更新,避免频繁下载安装包重装。
%======================================================================================
\section{数据库}
%======================================================================================
% 详细说明项目相关的数据库方面的需求。
%======================================================================================
    采用Oracle18.3数据库系统,合理设计数据库系统,要求能进行数据库的建立,调优,重组,重构,
    安全管控,报错问题的分析、汇总和处理、日常备份。

\section{\color{red} 数据库细化设计}
{\color{red}

\subsection{\color{red} 检索速度}
随着使用时间的增加,聊天记录,文件的数目都会快速增长,从而导致数据库检索时间减慢。为了保证检索文件,聊天记录等的速度,针对以下问题进行细化。


\subsubsection{\color{red} 机器配置}
\begin{itemize}
    \item 硬件要求:每个城市均配备三级存储。高速存储DRAM,中速存储DRAM,大容量存储机械硬盘。
    \item 软件要求: Oracle, Hbase
    \item 操作系统:Oracle 12c
\end{itemize}
\subsubsection{\color{red} 数据库选取}
即时通讯系统中,既包含传统的文本数据,可以用关系数据库存储;同时也包含音视频等非关系模式的数据,因此必须结合NoSQL数据库进行存储。\\
\\
关系数据库的查找速度较快,但不适于存储多媒体数据;\\
NoSQL的存取速度较慢,但具有多种新型拓扑结构,也适用于分布式存储。\\
\\
综合考虑,使用功能稳定,机密性强,兼容性好的Oracle18.3存储关系型数据,具有较完善安全机制的Hbase存储非关系型数据。

\subsubsection{\color{red} 存储方案}
为了提高访问速度,优化性能,采用以下方案:\\
1. RAID阵列存储 \\
其优势在于:\\
\begin{itemize}
    \item 逻辑统一:把多个磁盘组织在一起作为一个逻辑卷提供磁盘跨越功能;
    \item 加快速度:把数据分成多个数据块(Block)并行写入/读出多个磁盘以提高访问磁盘的速度
    \item 容错能力强:通过镜像或校验操作提供容错能力
\end{itemize}
2. 分级存储 \\
综合考虑性能与成本,采用分级存储方案:
在每1000个用户的密度下设置高性能存储器,使用SRAM, 容量小,存取速度快;而且由于
距离用户空间距离近,传输时间短。该SRAM仅存储1天内的数据。\\
在每100,000个用户的密度下设置中级性能存储器,其容量稍大,但存取稍慢。其存储近1个月的数据。\\
在主要城市设置大型存储机群。其存储该城市所有用户的本地数据,但存取较慢。\\
3. 三副本 \\
三副本可以提高并发访问的速度,且可以防止数据丢失。\\
采用异地存储。两个副本存储在本城市,提高访问速度;另一个副本云存储在外地,从而在本地出现灾害时,可以保证数据不丢失。

\subsubsection{\color{red} 建立索引}

创建索引可以大大提高系统的性能。例如,其可以大大加快数据的检索速度,加速表和表之间的连接,特别是在实现数据的参考完整性方面特别有意义;在使用分组和排序子句进行数据检索时,可以显著减少查询中分组和排序的时间;通过使用索引,可以在查询的过程中,使用优化隐藏器,提高系统的性能。\\

采用以下索引方式:
\begin{itemize}
    \item 聚集索引:一个表只能包含一个聚集索引。与非聚集索引相比,聚集索引通常提供更快的数据访问速度。因此,采用该表中最常使用的字段作为聚集索引的关键字。
    \item 主键索引: 数据库将自动设置主键索引。其一般为对应表项的ID。
\end{itemize}

\subsection{\color{red} 数据一致性}
在即时通讯系统中,并发用户很多。尤其是在线文档协作平台等功能,存在大量并发操作。
往往需要加锁。因此,需要对数据库设置合适的隔离级别,从而保证数据的一致性。
事务的隔离级别分为4种,其对于一致性的保证顺序为:\\
未提交读 < 提交读 < 可重复读 < 可串行读\\
一致性级别越高,访问速度也越低。\\

本数据库采用的方式为:根据流量自动调整隔离级别:\\
并行访问量 < 1000: 可串行读。其不存在脏读,不可重复读,幻像问题。\\
并行访问量 <100,000: 可重复读。其可能存在幻像问题。\\
并行访问量 >100,000: 提交读。其可能存在不可重复读和幻像问题。\\

三个级别都可以保证不出现脏读问题。其通过对锁的类型的切换来实现切换隔离级别。
}
\section{操作}
%======================================================================================
% 详细说明用户通常的和特殊的操作需求。
%======================================================================================
\noindent
用户基本操作方式:\\
    1. 标准鼠标键盘\\
    2. 触屏\\
    3. 语音控制\\
用户核心操作支持:\\
    1. 文字信息处理发送\\
    2. 图片文件编辑发送\\
    3. 音视频录制发送\\
用户其他常用操作支持:  \\
    1. 浏览Board信息\\
    2. 添加删除好友,加群退群\\
    3. 查看隐私内容或安全属性\\
    4. 管理日历\\
    5. 通过邮箱接口管理邮件\\
用户特殊操作支持:\\
    1. 自动更新\\
    2. 卸载\\
    3. 报告错误\\
    4. 修改隐私内容或安全属性
\section{本地化}
%======================================================================================
% 描述支持多语种的需求。
%======================================================================================
支持中文(简体)、中文(繁体)、英文、法文,俄文,阿拉伯文。