\chapter{可行性分析结果}
%======================================================================================
% 描述对分配需求的可行性分析结果。
%======================================================================================
\section{技术可行性}
%======================================================================================
    技术可行性分析主要分析现有技术条件能否顺利完成开发工作,软硬件配置能否满足
    开发者需求。即时通讯系统的主要功能是提供现代工作场景下信息交互,通知发布,团队组织、
    任务管理的平台解决方案。所依赖的的安卓、IOS、Windows、Linux操作系统,Intel和智能移动端硬件平台,JAVA开发语言和
    Eclipse开发环境都是成熟的开发生态圈,技术上可行。
%======================================================================================
\section{经济可行性}
%======================================================================================
    经济可行性上,本团队人力成本相对大型互联网企业偏低,场地使用学校实验室不需
    要支付额外的费用,开发除了硬件消耗,服务器租用,软件维护的少量费用以外,
    不存在其余开发费用。启动资金在个人承受范围以内,经营成本相对低。通过广告、
    融资之后硬件和人力经济压力也会有所缓解,因此经济上具有可行性。
%======================================================================================
\section{法律可行性}
%======================================================================================
\begin{itemize}
    \item 侵犯专利权:无
    \item 侵犯版权:无
\end{itemize}
%======================================================================================
\section{可行性分析结论}
%======================================================================================
    随着现代社会发展与互联网产业崛起,高等院校、科研机构、企业部门和开发团队对于即时通讯平台的功能和
    性能需求与日俱增,需要一个保证即时通讯效率,提供不同团队模式合作、场景应用、信息发布,任务管理,文档管理
    与在线合作功能的解决方案。根据以上分析可知,开发面向工作团队的即时通讯系统有着稳定可靠的技术、经济和
    法律可行性,将带来巨大的社会价值和良好的商业应用前景。所以,我们认为开发这个平台的条件已经具备,可以开始开发工作。
%======================================================================================