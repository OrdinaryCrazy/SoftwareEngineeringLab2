\chapter{具体需求}
\section{功能需求}
%====================================================================================================================
% 本子章节应描述软件产品的输入怎样被转换成输出。它描述了软件必须执行的基本动作。 
% 对每一类功能或有时对每一个单独的功能,必须描述输入、处理、输出方面的需求。这些通常以下面四个子段落来组织:
% \subsubsection{介绍}
% 逐条列出与本特性相关的功能需求。包括项目如何响应预期的错误输入,非法条件和无效输入。需求应该简明,完整,不含糊,可验证,必要的。 
% 当需要的信息不确定的时候使用“待定”。
% \subsubsection{输入}
% 本子段落应包含下列内容:
% A. 对该功能所有输入数据的详细描述,包括:
%	输入来源、数量、度量单位、时间要求、包含精度和容忍度的有效输入范围	
% B. 在适当的地方提供的对接口规格或接口控制文档的参考。
% \subsubsection{处理}
% 本子段落应描述对输入数据所执行的所有操作和如何获得输出的过程。这包括下列规格:
% A. 输入数据的有效性检测。
% B. 操作的确切次序,包括各事件的时序。
% C. 对异常情况的回应,例如:溢出、通信失败、错误处理
% D. 用于把系统输入转换到相应输出的任何方法(诸如方程式,数学算法,逻辑操作)。例如,这可能描述下列方面:
%	对工资单里代扣所得税的计算公式、用于气象预报的气象模型。	
% E. 对输出数据的有效性检测。
% \subsubsection{输出}
% 本子段落应包含:
% A. 对该功能所有输出数据的详细描述,这个描述包括:
%		输出的到何处(如打印机,文件)、数量、度量单位、时序、包含精确度和容忍度的有效输出范围
%		对非法值的处理、错误消息	
% B. 在适当的地方提供对接口规格或接口控制文档的参考。
% 此外,对那些需求集中在输入/输出行为的系统,SRS应描述所有重要的输入/输出行为及输入输出对的次序。
% 对一个需要记忆其行为以根据输入和过去的行为进行反应的系统,输入输出对的次序是要求的;这种功能行为就类似于有限状态机。
%====================================================================================================================
% 王浩宇
\subsection{一对一即时通讯}
\subsubsection{介绍}
\subsubsection{输入}
\subsubsection{处理}
\subsubsection{输出}
% 王浩宇
\subsection{R.INTF.CALC.002: 多情境群聊功能}
多种群聊场景支持:课程、班级、工程团队、工作小组
\subsubsection{介绍}
\subsubsection{输入}
\subsubsection{处理}
\subsubsection{输出}
% 戴路
\subsection{R.INTF.CALC.003: 活动/任务发布与管理功能}
\subsubsection{介绍}
\subsubsection{输入}
\subsubsection{处理}
\subsubsection{输出}
% 王浩宇
\subsection{R.INTF.CALC.004: 音视频通话(会议)功能}
\subsubsection{介绍}
\subsubsection{输入}
\subsubsection{处理}
\subsubsection{输出}
\subsection{R.INTF.CALC.005: 通讯录功能}
在日常使用中,我们一般不会和陌生人通信,而是和好友进行联系,或是在加入的群聊中发言。于是,我们使用通讯录管理所有的好友和群聊。
\subsubsection{介绍}
对用户而言,该功能的需求为:
\begin{itemize}
  \item 用户可以添加其他人为好友,可以通过账号查询、二维码等多种方式添加好友,向其发送好友请求。
  \item 用户可以加入群聊,可以通过账号查询、二维码等多种方式加入群聊,向其发送入群申请。
  \item 用户可以接受或拒绝其他用户的好友请求和入群邀请。
  \item 用户可以在通讯录中查询好友或群聊,进行通讯。
  \item 用户可以单方面删除好友。
  \item 用户可以退出群聊。
  \item 用户可以为好友进行分组。
  \item 用户可以将其他人加入或移出黑名单,用户不会接收来自黑名单的任何信息。
  \item 用户可以为好友增加备注信息。
\end{itemize}
\subsubsection{输入}
用户对好友的添加、删除和查询,用户设置的备注信息。
\subsubsection{处理}
\begin{itemize}
  \item \textbf{添加好友(群聊):} 可以根据提供的信息定位到具体的用户(群聊),向他发送请求。在他接受后,把他的信息加入通讯录。
  \item \textbf{删除好友:} 从用户的通讯录中删除该好友,同时从他的通讯录中删除用户信息。
  \item \textbf{退出群聊:} 从用户的通讯录中删除该群聊,同时从它的群成员中删除用户信息。
  \item \textbf{查询好友(群聊):} 从通讯录中找到对应的好友(群聊),可以采用遍历、二分查找、哈希等方法。
  \item \textbf{备注好友:} 为好友增加备注。
  \item \textbf{好友分组:} 在通讯录中对好友分组管理。
  \item \textbf{黑名单:} 屏蔽来自黑名单的所有信息。
\end{itemize}
\subsubsection{输出}
展示用户在上述操作后的新通讯录。

\subsection{R.INTF.CALC.006: 聊天记录功能}
在很多时候,用户需要之前的聊天内容。于是,我们提供了聊天记录功能。
\subsubsection{介绍}
对用户而言,该功能的需求为:
\begin{itemize}
  \item 允许用户查询聊天记录,且保持设备间同步。
  \item 允许用户导出聊天记录(txt格式、xls格式等)。
\end{itemize}
\subsubsection{输入}
对聊天记录的查询条件(时间、聊天对象、关键词等),以及导出命令和格式。
\subsubsection{处理}
\begin{itemize}
  \item 在用户聊天的同时,记录所有的聊天内容,存放在聊天记录中(聊天记录存储在云端)。
  \item 对用户提供的查询条件,检索聊天记录,返回对应的内容。
  \item 在导出时,将聊天记录以对应的格式写入文件。
\end{itemize}
\subsubsection{输出}
\begin{itemize}
  \item 向用户展示查询得到的聊天内容。
  \item 导出操作会输出一个写有聊天记录的文件。如果文件写入失败(目录不存在、存储已满等),向用户报错。
\end{itemize}

% 王浩宇
\subsection{消息提醒功能(R.INTF.CALC.007)}
\subsubsection{介绍}
\subsubsection{输入}
\subsubsection{处理}
\subsubsection{输出}
% 戴路
\subsection{Board(广场)功能(R.INTF.CALC.008)}
\subsubsection{介绍}
\subsubsection{输入}
\subsubsection{处理}
\subsubsection{输出}
\subsection{R.INTF.CALC.009: 个性化好友推荐功能}
为了扩大社交圈,用户需要更多的好友。于是,我们设计了该功能,为用户推荐志趣相投的好友。
\subsubsection{介绍}
对用户而言,该功能的需求为:
\begin{itemize}
  \item 用户可以输入一定的条件,系统会根据这些条件个性化地推荐好友
  \item 用户可以向推荐的好友发送好友请求
\end{itemize}
\subsubsection{输入}
好友选择条件。
\subsubsection{处理}
\begin{itemize}
  \item 根据用户平时的聊天信息、好友信息等,利用推荐算法得到一个用户集合。
  \item 根据用户的好友选择条件,从该集合中选择符合要求的用户子集。
  \item 从该子集中随机向用户输出五个符合要求的用户作为推荐的好友。
\end{itemize}
\subsubsection{输出}
向用户展示系统推荐的好友。
% 戴路
\subsection{在线文档协作平台}
\subsubsection{介绍}
\subsubsection{输入}
\subsubsection{处理}
\subsubsection{输出}
% 戴路
\subsection{账号保护 & 隐私保护}
\subsubsection{介绍}
\subsubsection{输入}
\subsubsection{处理}
\subsubsection{输出}
% 王浩宇
\subsection{R.INTF.CALC.012: 日历管理功能}
\subsubsection{介绍}
\subsubsection{输入}
\subsubsection{处理}
\subsubsection{输出}
% 王浩宇
\subsection{个人本地和云端文件管理}
\subsubsection{介绍}
\subsubsection{输入}
\subsubsection{处理}
\subsubsection{输出}
\subsection{R.INTF.CALC.014: 邮箱接口功能}
邮箱在日常生活中被广泛应用。于是,我们提供了邮箱接口功能。
\subsubsection{介绍}
对用户而言,该功能的需求为:
\begin{itemize}
  \item 用户可以为自己的账户绑定一个邮箱。
  \item 一旦邮箱收到新的邮件,就会给用户以消息提醒。
  \item 用户点击邮件按钮后,可以撰写邮件,并通过绑定的邮箱发送。
\end{itemize}
\subsubsection{输入}
用户绑定的邮箱,以及邮件信息。
\subsubsection{处理}
\begin{itemize}
  \item 在账户信息中增加一个条目,存储绑定的电子邮箱。
  \item 查询绑定的电子邮箱是否有邮件,一旦有邮件就调用消息提醒接口进行提醒。
  \item 读取用户输入的邮件信息,使用绑定的邮箱发送。
\end{itemize}
\subsubsection{输出}
使用消息提醒接口进行消息提醒。

\section{性能需求}
%====================================================================================================================
% 如果有性能方面的需求,在这里列出并解释他们的原理。以帮助开发者理解意图以做出正确的设计选择。在实时系统中的时序关系。保证需求尽可能的详细而精确。
%====================================================================================================================
\subsection{总体性能需求}

\subsubsection{支持的终端数目}
\subsubsection{支持的同时使用的用户数目}
\subsubsection{处理的文件和记录的数目}
\subsubsection{表和文件的大小}
\subsubsection{同时处理的事务数量}
\subsubsection{正常信息发送延迟}
\subsubsection{正常操作响应时间}
\subsubsection{平台适应性}
\subsubsection{内存占用限制}
\subsubsection{正常工作状态耗电限制}

\subsection{具体功能的性能需求}
% \subsubsection{一对一即时通讯}
% \subsubsection{群聊}
% \subsubsection{活动/任务发布与管理}
%====================================================================================================================
% 本子章节应从整体上描述静态和动态的量化的对软件(或人与软件交互)的需求。
% 静态的量化需求可能包括:
%	A. 支持的终端数目。
%	B. 支持的同时使用的用户数目。
%	C. 处理的文件和记录的数目。
%	D. 表和文件的大小。
% 动态的量化需求可能包括:
%	A. 在正常和峰值工作量条件下特定时间段(如一小时)
%	B. 处理的事务和任务的数目以及数据量。
% 所有的这些需求应以可测量的术语进行描述,例如所有的操作应在1秒内被处理完成,而不是描述成操作员不必等待操作的完成。
% 注意: 用于一个具体功能的量化限制通常在该功能的处理子章节中描述。
%====================================================================================================================
\section{外部接口需求}
\subsection{用户接口}
%====================================================================================================================
% 详细描述系统与用户之间的接口
% 这应描述下述内容:
% A. 对每种人机界面,软件所必须支持的特性。例如,如果系统用户通过一个显示终端进行操作,那么应包含下述内容:
% 	要求的屏幕格式、页面规划及报告或菜单的内容
%	输入和输出的相关时序、一些组合功能键的用法
% B. 与系统用户接口使用相关的所有方面。这可能只是一个简单的关于系统怎样展示给用户而该做什么和不该做什么的列表。
%	例如提供关于长或短错误消息选项。和所有其它需求一样,这些需求也应能被检验,
%	例如,四级打字员经一小时的培训后能在Z分钟内完成功能X,而不是一个打字员能完成功能X。
%====================================================================================================================
\subsection{软件接口}
%====================================================================================================================
% 详细描述与其他系统 /模块 /项目之间的接口
% 在此应描述如何使用其它(必需的)软件产品(例如,数据管理系统,操作系统,或算法工具包),
% 以及与其它应用系统的接口(例如,协议处理系统和数据库管理系统之间的接口)。
% 对每个必需的软件产品,应提供下列信息:
%	A. 名字、B. 助记符、C. 版本号、 D. 来源
% 对每个接口,本部分应:
%	A .	讨论与本软件产品相关的接口软件的目的。
%	B.	按消息/函数内容和格式定义接口。如果接口已在其它文档中很清楚地描述,就没有必要在这儿进行详细描述,但需说明应参考的文档。
%====================================================================================================================
\subsection{硬件接口}
 %====================================================================================================================
% 详细描述与硬件的接口
% 在此描述软件产品和系统硬件组件之间接口的逻辑特征,也包括支持哪些设备、怎样支持这些设备和协议等。
% 按软/硬件协议内容和格式定义接口。如果接口已在其它文档中很清楚地描述,就没有必要在这儿进行详细描述,但需说明应参考的文档。
%====================================================================================================================
\subsection{通讯接口}
%====================================================================================================================
% 详细描述通讯接口,如本地网络协议等。
% 按消息/函数内容和格式定义接口。如果接口已在其它文档中很清楚地描述,就没有必要在这儿进行详细描述,但需说明应参考的文档。
%====================================================================================================================
