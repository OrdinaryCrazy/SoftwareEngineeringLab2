\chapter{具体需求}
% <The following sections must be repeated for each requirement. >
% 在每一条需求描述中重复下列部分
\section{功能需求}
% 本子章节应描述软件产品的输入怎样被转换成输出。它描述了软件必须执行的基本动作。 
% 对每一类功能或有时对每一个单独的功能,必须描述输入、处理、输出方面的需求。这些通常以下面四个子段落来组织:
% 王浩宇
\subsection{一对一即时通讯}
\subsubsection{介绍}
\subsubsection{输入}
\subsubsection{处理}
\subsubsection{输出}
% 王浩宇
\subsection{R.INTF.CALC.002: 多情境群聊功能}
多种群聊场景支持:课程、班级、工程团队、工作小组
\subsubsection{介绍}
\subsubsection{输入}
\subsubsection{处理}
\subsubsection{输出}
% 戴路
\subsection{R.INTF.CALC.003: 活动/任务发布与管理功能}
\subsubsection{介绍}
\subsubsection{输入}
\subsubsection{处理}
\subsubsection{输出}
% 王浩宇
\subsection{R.INTF.CALC.004: 音视频通话(会议)功能}
\subsubsection{介绍}
\subsubsection{输入}
\subsubsection{处理}
\subsubsection{输出}
\subsection{R.INTF.CALC.005: 通讯录功能}
在日常使用中,我们一般不会和陌生人通信,而是和好友进行联系,或是在加入的群聊中发言。于是,我们使用通讯录管理所有的好友和群聊。
\subsubsection{介绍}
对用户而言,该功能的需求为:
\begin{itemize}
  \item 用户可以添加其他人为好友,可以通过账号查询、二维码等多种方式添加好友,向其发送好友请求。
  \item 用户可以加入群聊,可以通过账号查询、二维码等多种方式加入群聊,向其发送入群申请。
  \item 用户可以接受或拒绝其他用户的好友请求和入群邀请。
  \item 用户可以在通讯录中查询好友或群聊,进行通讯。
  \item 用户可以单方面删除好友。
  \item 用户可以退出群聊。
  \item 用户可以为好友进行分组。
  \item 用户可以将其他人加入或移出黑名单,用户不会接收来自黑名单的任何信息。
  \item 用户可以为好友增加备注信息。
\end{itemize}
\subsubsection{输入}
用户对好友的添加、删除和查询,用户设置的备注信息。
\subsubsection{处理}
\begin{itemize}
  \item \textbf{添加好友(群聊):} 可以根据提供的信息定位到具体的用户(群聊),向他发送请求。在他接受后,把他的信息加入通讯录。
  \item \textbf{删除好友:} 从用户的通讯录中删除该好友,同时从他的通讯录中删除用户信息。
  \item \textbf{退出群聊:} 从用户的通讯录中删除该群聊,同时从它的群成员中删除用户信息。
  \item \textbf{查询好友(群聊):} 从通讯录中找到对应的好友(群聊),可以采用遍历、二分查找、哈希等方法。
  \item \textbf{备注好友:} 为好友增加备注。
  \item \textbf{好友分组:} 在通讯录中对好友分组管理。
  \item \textbf{黑名单:} 屏蔽来自黑名单的所有信息。
\end{itemize}
\subsubsection{输出}
展示用户在上述操作后的新通讯录。

\subsection{R.INTF.CALC.006: 聊天记录功能}
在很多时候,用户需要之前的聊天内容。于是,我们提供了聊天记录功能。
\subsubsection{介绍}
对用户而言,该功能的需求为:
\begin{itemize}
  \item 允许用户查询聊天记录,且保持设备间同步。
  \item 允许用户导出聊天记录(txt格式、xls格式等)。
\end{itemize}
\subsubsection{输入}
对聊天记录的查询条件(时间、聊天对象、关键词等),以及导出命令和格式。
\subsubsection{处理}
\begin{itemize}
  \item 在用户聊天的同时,记录所有的聊天内容,存放在聊天记录中(聊天记录存储在云端)。
  \item 对用户提供的查询条件,检索聊天记录,返回对应的内容。
  \item 在导出时,将聊天记录以对应的格式写入文件。
\end{itemize}
\subsubsection{输出}
\begin{itemize}
  \item 向用户展示查询得到的聊天内容。
  \item 导出操作会输出一个写有聊天记录的文件。如果文件写入失败(目录不存在、存储已满等),向用户报错。
\end{itemize}

% 王浩宇
\subsection{消息提醒功能(R.INTF.CALC.007)}
\subsubsection{介绍}
\subsubsection{输入}
\subsubsection{处理}
\subsubsection{输出}
% 戴路
\subsection{Board(广场)功能(R.INTF.CALC.008)}
\subsubsection{介绍}
\subsubsection{输入}
\subsubsection{处理}
\subsubsection{输出}
% R.INTF.CALC.001 Calculating expression
% R.INTF.CALC.002 Print
% 用需求编号加上简短词汇做为功能需求名,不要用“功能需求(1)”作为功能名,例如:R.INTF.CALC.001 计算表达式
% 需求编号规则按照软件需求管理规程(REP01)进行
% \subsubsection{介绍}
% 逐条列出与本特性相关的功能需求。包括项目如何响应预期的错误输入,非法条件和无效输入。需求应该简明,完整,不含糊,可验证,必要的。 当需要的信息不确定的时候使用“待定”。
% \subsubsection{输入}
% 本子段落应包含下列内容:
% A. 对该功能所有输入数据的详细描述,包括:
%		输入来源
%		数量
%		度量单位
%		时间要求
%		包含精度和容忍度的有效输入范围	
% B. 在适当的地方提供的对接口规格或接口控制文档的参考。
% \subsubsection{处理}
% 本子段落应描述对输入数据所执行的所有操作和如何获得输出的过程。这包括下列规格:
% A. 输入数据的有效性检测。
% B. 操作的确切次序,包括各事件的时序。
% C. 对异常情况的回应,例如:
%		溢出
%		通信失败
%		错误处理
% D. 用于把系统输入转换到相应输出的任何方法(诸如方程式,数学算法,逻辑操作)。例如,这可能描述下列方面:
%		对工资单里代扣所得税的计算公式。
%		用于气象预报的气象模型。	
% E.	对输出数据的有效性检测。
% \subsubsection{输出}
% 本子段落应包含:
% A. 对该功能所有输出数据的详细描述,这个描述包括:
%		输出的到何处(如打印机,文件)
%		数量
%		度量单位
%		时序
%		包含精确度和容忍度的有效输出范围
%		对非法值的处理
%		错误消息	
% B. 在适当的地方提供对接口规格或接口控制文档的参考。
% 此外,对那些需求集中在输入/输出行为的系统,SRS应描述所有重要的输入/输出行为及输入输出对的次序。对一个需要记忆其行为以根据输入和过去的行为进行反应的系统,输入输出对的次序是要求的;这种功能行为就类似于有限状态机。
\section{性能需求}
<If there are performance requirements, state them here and explain their rationale, to help the developers understand the intent and make suitable design choices. Specifies the timing relationships for real time systems. Such requirements should be made as specific as possible. >

如果有性能方面的需求,在这里列出并解释他们的原理。以帮助开发者理解意图以做出正确的设计选择。在实时系统中的时序关系。保证需求尽可能的详细而精确。
\subsection{性能需求1}
本子章节应从整体上描述静态和动态的量化的对软件(或人与软件交互)的需求。

静态的量化需求可能包括:

A. 支持的终端数目。

B. 支持的同时使用的用户数目。

C. 处理的文件和记录的数目。

D. 表和文件的大小。

动态的量化需求可能包括:

A. 在正常和峰值工作量条件下特定时间段(如一小时)

B. 处理的事务和任务的数目以及数据量。

所有的这些需求应以可测量的术语进行描述,例如所有的操作应在1秒内被处理完成,而不是描述成操作员不必等待操作的完成。

注意: 用于一个具体功能的量化限制通常在该功能的处理子章节中描述。
\section{外部接口需求}
\subsection{用户接口}
<The interface of the system with the User and vice versa should be explained in detail. >

详细描述系统与用户之间的接口

This section should include:
A. Features that must be supported by the software for eachman-machine interface. For example, if the user operates from a display terminal, then the following should be included:
		Screen format required
		Page layout and content of report and menu
		Timing sequence for input and output
		Usage of some functional key combinations
B. Every aspect about the use of the system's user interface. It could be a list that shows the user what should do and what should not do.  For example, an option of overlong or overshort message. . And same as other requirements, these requirements should be easily verified. For example, saying "A level 4 typist can finish function X in Z minutes after a one-hour training." instead of "A typist can finish function X"	

这应描述下述内容:

A. 对每种人机界面,软件所必须支持的特性。例如,如果系统用户通过一个显示终端进行操作,那么应包含下述内容:
要求的屏幕格式
页面规划及报告或菜单的内容
输入和输出的相关时序
一些组合功能键的用法

B. 与系统用户接口使用相关的所有方面。这可能只是一个简单的关于系统怎样展示给用户而该做什么和不该做什么的列表。例如提供关于长或短错误消息选项。和所有其它需求一样,这些需求也应能被检验,例如,四级打字员经一小时的培训后能在Z分钟内完成功能X,而不是一个打字员能完成功能X。

\subsection{软件接口}
<The interface with other system/modules/projects should be explained in detail. >

详细描述与其他系统 /模块 /项目之间的接口

Describes how to use the other (required) software products. (such as data management system, operation system, or algorithm tools package), and the interfaces to other application systems (such as interfaces between the protocol process system and the database management system )
For each required software product, following information should be provided:
A. Name
B. Mnemonic symbol
C. Version number
D. Source
For each interface, this section should:
A. Discuss the objective of the required software.
B. Define the interfaces by content and format of message/function. If the interfaces have been clearly described in other documents, it is not necessary to describe in detail here. But the reference of those documents should be given.

在此应描述如何使用其它(必需的)软件产品(例如,数据管理系统,操作系统,或算法工具包),以及与其它应用系统的接口(例如,协议处理系统和数据库管理系统之间的接口)。

对每个必需的软件产品,应提供下列信息:
A.	名字
B.	助记符
C.	版本号
D.	来源

对每个接口,本部分应:

A .	讨论与本软件产品相关的接口软件的目的。

B.	按消息/函数内容和格式定义接口。如果接口已在其它文档中很清楚地描述,就没有必要在这儿进行详细描述,但需说明应参考的文档。

\subsection{硬件接口}
<The interface with other hardware components should be explained in detail. >

详细描述与硬件的接口

Describes the logical features of the interface between the software and hardware components, including the equipment supported and how the equipment and protocol is supported. 

Defines the interfaces according to the content and format of the software/hardware protocol. If the interfaces have been clearly described in other documents, it is not necessary to describe in detail here. But the reference of those documents should be given.

在此描述软件产品和系统硬件组件之间接口的逻辑特征,也包括支持哪些设备、怎样支持这些设备和协议等。
 
按软/硬件协议内容和格式定义接口。如果接口已在其它文档中很清楚地描述,就没有必要在这儿进行详细描述,但需说明应参考的文档。

\subsection{通讯接口}
<This should specify the various interfaces to communications such as local network protocols, etc.>

详细描述通讯接口,如本地网络协议等。

Defines the interfaces according to the content and format of the message/function. If the interfaces have been clearly described in other documents, it is not necessary to describe in detail here. But the reference of those documents should be given.

按消息/函数内容和格式定义接口。如果接口已在其它文档中很清楚地描述,就没有必要在这儿进行详细描述,但需说明应参考的文档。
