\chapter{\color{red} 安全保密设计}
%===================================================================
% 可能的内容包括保密性、是否采取加密传输、密钥如何分发和管理等。
%===================================================================
    \section{\color{red} 保密性}
    对以下信息在客户端和服务器加密保存,单次使用后立即销毁解密副本:
    \begin{itemize}
        \item 用户信息文件
        \item 用户登录、注册、注销、密码修改信息
        \item 用户安全问题与相关设置信息
        \item 用户操作日志
        \item 用户日历
        \item 用户通讯录
        \item 用户聊天记录
        \item 用户Board日志
        \item 在线文档
        \item 用户邮件
        \item 审批记录
        \item 审批材料
    \end{itemize}
%===================================================================
    \section{加密传输}
    \subsection{RSA算法加密小型数据}
    对客户端和服务器、客户端和客户端之间的小型数据传输,适合采用RSA加密。\\
    例如: 用户名,密码, 验证信息,保密问题等\\
    RSA算法的流程为:
\begin{itemize}
    

 \item 公钥与私钥的产生 \\

1. 选择两个大素数$ {\displaystyle p}$和$ {\displaystyle {q}}$, ${\displaystyle{p}}$ 不等于 ${\displaystyle {q}}$,计算 ${\displaystyle {N=pq}}$。\\
2. 根据欧拉函数,求得$ {\displaystyle r=\varphi (N)=\varphi (p)\varphi (q)=(p-1)(q-1)} $, e与r互质\\
3. 选择一个小于$ {\displaystyle r} $的整数$ {\displaystyle e} $,使 ${\displaystyle e} $与 ${\displaystyle r}$ 互质。\\
4. 并求 ${\displaystyle d} $使得$ {\displaystyle ed\equiv 1{\pmod {r}}}$ 。\\
5. ${\displaystyle (N,e)}$是公钥,${\displaystyle (N,d)}$是私钥。

\item 加密消息\\
加密过程为:\\
${\displaystyle c\equiv n^{e}{\pmod {N}}}$

\item 解密消息 \\
${\displaystyle n\equiv c^{d}\ (\mathrm {mod} \ N)} $
    \end{itemize}
    \subsection{DES算法加密文件等大型数据}
    即时通讯系统中,文件,用户信息,在线文档等大型数据适合采用DES加密。
    DES为对称加密算法。
    与RSA加密算法相比,AES加密算法的优点为加解密的速度更快、加密强度最高、且不占用硬件资源。其更适合加密大型数据。其算法主要分为两步:
    \begin{itemize}
\item 1. 初始置换\\
其功能是把输入的64位数据块按位重新组合,并把输出分为L0、R0两部分,每部分各长32位,其置换规则为将输入的第58位换到第一位,第50位换到第2位……依此类推。
\item 2. 逆置换\\
经过16次迭代运算后,得到L16、R16,将此作为输入,进行逆置换,逆置换正好是初始置换的逆运算,由此即得到密文输出。
     \end{itemize}
%===================================================================
    \section{人机认证与安全问题}
    在登录、注册、修改密码、绑定邮箱等敏感操作的界面添加人机认证(图形/短信验证码,手动滑块)并要求用户回答个人信息和安全设置中的安全问题。
%===================================================================