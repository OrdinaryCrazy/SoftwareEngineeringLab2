\chapter{安全保密设计}
%===================================================================
% 可能的内容包括保密性、是否采取加密传输、密钥如何分发和管理等。
%===================================================================
    \section{保密性}
    对以下信息在客户端和服务器加密保存,单次使用后立即销毁解密副本:
    \begin{itemize}
        \item 用户信息文件
        \item 用户登录、注册、注销、密码修改信息
        \item 用户安全问题与相关设置信息
        \item 用户操作日志
        \item 用户日历
        \item 用户通讯录
        \item 用户聊天记录
        \item 用户Board日志
        \item 在线文档
        \item 用户邮件
    \end{itemize}
%===================================================================
    \section{加密传输}
    \subsection{RSA算法加密小型数据}
    对客户端和服务器、客户端和客户端之间的小型数据传输,适合采用RSA加密。\\
    例如: 用户名,密码, 验证信息,保密问题等\\
    RSA算法的流程为:
    公钥与私钥的产生
假设Alice想要通过一个不可靠的媒体接收Bob的一条私人消息。她可以用以下的方式来产生一个公钥和一个私钥:

随意选择两个大的素数 {\displaystyle {p}} 和 {\displaystyle {q}}, {\displaystyle{}p} 不等于 {\displaystyle q},计算 {\displaystyle N=pq}。\\
根据欧拉函数,求得 {\displaystyle r=\varphi (N)=\varphi (p)\varphi (q)=(p-1)(q-1)} \\
选择一个小于 {\displaystyle r} 的整数 {\displaystyle e} ,使 {\displaystyle e} 与 {\displaystyle r} 互质。并求得 {\displaystyle e} 关于 {\displaystyle r} 的模逆元,命名为 {\displaystyle d} (求 {\displaystyle d} 令 {\displaystyle ed\equiv 1{\pmod {r}}} )。(模逆元存在,当且仅当 {\displaystyle e} 与 {\displaystyle r} 互质)
将 {\displaystyle p} 和 {\displaystyle q} 的记录销毁。\\
{\displaystyle (N,e)}是公钥, {\displaystyle (N,d)}是私钥。Alice将她的公钥 {\displaystyle (N,e)} 传给Bob,而将她的私钥 {\displaystyle (N,d)}藏起来。

加密消息\\
假设Bob想给Alice送一个消息 {\displaystyle m} ,他知道Alice产生的 {\displaystyle N} 和 {\displaystyle e} 。他使用起先与Alice约好的格式将 {\displaystyle m} 转换为一个小于 {\displaystyle N} 的非负整数 {\displaystyle n} ,比如他可以将每一个字转换为这个字的Unicode码,然后将这些数字连在一起组成一个数字。假如他的信息非常长的话,他可以将这个信息分为几段,然后将每一段转换为 {\displaystyle n} 。用下面这个公式他可以将 {\displaystyle n} 加密为 {\displaystyle c} :

{\displaystyle c\equiv n^{e}{\pmod {N}}}
计算 {\displaystyle c} 并不复杂。Bob算出 {\displaystyle c} c后就可以将它传递给Alice。

解密消息
Alice得到Bob的消息 {\displaystyle c} c后就可以利用她的密钥 {\displaystyle d} d来解码。她可以用以下这个公式来将 {\displaystyle c} c转换为 {\displaystyle n} n:

{\displaystyle n\equiv c^{d}\ (\mathrm {mod} \ N)} {\displaystyle n\equiv c^{d}\ (\mathrm {mod} \ N)}
得到 {\displaystyle n} n后,她可以将原来的信息 {\displaystyle m} m重新复原。

解码的原理是

{\displaystyle c^{d}\equiv n^{e\cdot d}\ (\mathrm {mod} \ N)}  c^d \equiv n^{e \cdot d}\ (\mathrm{mod}\ N)
已知 {\displaystyle ed\equiv 1{\pmod {r}}} {\displaystyle ed\equiv 1{\pmod {r}}},即 {\displaystyle ed=1+h\varphi (N)} {\displaystyle ed=1+h\varphi (N)}。 由欧拉定理得:

{\displaystyle n^{ed}=n^{1+h\varphi (N)}=n\left(n^{\varphi (N)}\right)^{h}\equiv n(1)^{h}{\pmod {N}}\equiv n{\pmod {N}}} {\displaystyle n^{ed}=n^{1+h\varphi (N)}=n\left(n^{\varphi (N)}\right)^{h}\equiv n(1)^{h}{\pmod {N}}\equiv n{\pmod {N}}}
    
    \subsection{}
%===================================================================
    \section{人机认证与安全问题}
    在登录、注册、修改密码、绑定邮箱等敏感操作的界面添加人机认证(图形/短信验证码,手动滑块)并要求用户回答个人信息和安全设置中的安全问题。
%===================================================================