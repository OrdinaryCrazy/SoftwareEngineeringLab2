\chapter{\color{red} 接口设计}
%====================================================================================
\section{外部接口}
%====================================================================================
% 比如说需要用到支付宝等外部支付系统,接口应当如何封装。
%====================================================================================
JavaMail是由Sun发布的用来处理email的API,它是Java EE的一部分。我们调用该接口查询和发送邮件。


\section{\color{red} 内部接口}
{\color{red}
\subsection{\color{red}状态码(返回值)定义}
\begin{itemize}
    \item 001: 操作执行成功
    \item 002: 数据不存在
    \item 003: 网络连接错误
    \item 004: 数据不一致
    \item 005: 访问冲突
    \item 006:输入数据错误
    \item 007: 访问超时
\end{itemize}
}


\subsection{\color{red} 接口中的数据结构定义}
\subsubsection{\color{red} 数据结构1:User\_server}
该数据结构定义了用户的基本属性。
\begin{lstlisting}[language=Java, caption=User定义]
public class User{
    private String ID;//用户ID,唯一
    private String name;//昵称,不唯一
    private String sex="Male";//性别,默认男性
    private Date birthday;//用户的出生日期,可以为空
    private Time createTime;//账户创建时间
    private String address;//用户地址,可以为空
    private String email;//用户绑定的邮箱,可以为空
    private ArrayList<String> tpIDs; //用户的第三方平台账号
    private ArrayList<String> tpPasswords; //用户的第三方平台密码
}
\end{lstlisting}

\subsubsection{\color{red} 数据结构2:Message\_server}
该数据结构定义了一对一聊天时发送的信息,所有属性都是不可修改的。此外,在发送文件时,该数据结构也会被使用,用于声明文件的一些基本信息(使用message域)。
\begin{lstlisting}[language=Java, caption=Message定义]
public class Message{
    private User sender,receiver;//该信息的发送方和接收方
    private String message;//信息内容
    private Time sendtime;//发送时间
}
\end{lstlisting}

\subsubsection{\color{red} 数据结构3:GroupMessage\_server}
该数据结构定义了群聊时发送的信息,继承自Message类。同样,所有属性都是不可修改的。
\begin{lstlisting}[language=Java, caption=GroupMessage定义]
public class GroupMessage extends Message{
    private Group group;//所属的群聊
}
\end{lstlisting}

{\color{red}
\subsubsection{\color{red} 数据结构4:Group\_server}
该数据结构定义了群聊的基本属性。
\begin{lstlisting}[language=Java, caption=Group定义]
public class Group{
    private String ID;//群ID,唯一
    private String name;//群名称,不唯一
    private User master;//群主
    private User[] managerList;//管理员列表
    private User[] memberList;//群成员列表
    private Scene scene; //场景
    private ArrayList<Issue_server> issuelist;//本群中所有的issue
    
\end{lstlisting}
}


\subsubsection{\color{red} 数据结构5:Friend\_server}
该数据结构定义了通讯录中好友的基本属性。通讯录可以对好友分组管理,这是通过tag实现的,具有相同tag的用户被认为是一个组。
\begin{lstlisting}[language=Java, caption=Friend定义]
public class Friend{
    private User user;//用户
    private String tag;//分组标记,内容是组名
    private String remark;//备注名
}
\end{lstlisting}


\subsubsection{\color{red} 数据结构6:AddressBook\_server}
该数据结构定义了通讯录,包括好友列表,群聊列表和黑名单。
\begin{lstlisting}[language=Java, caption=AddressBook定义]
public class AddressBook{
    private Friend[] friendList;//好友列表
    private User[] blackList;//黑名单
    private Group[] groupList;//群聊列表
}
\end{lstlisting}

\subsubsection{\color{red} 数据结构7:Task\_server}
该数据结构定义了任务的基本属性,它将在活动/任务发布与管理功能中应用。
\begin{lstlisting}[language=Java, caption=Task定义]
public class Task{
    private User organizer;//组织者
    private User[] participants;//参与者列表
    private String address;//任务地点
    private Time time;//任务时间
    private String content;//任务内容
}
\end{lstlisting}

\subsubsection{\color{red} 数据结构8:Activity\_server}
该数据结构定义了活动项的基本属性,将被用于日历应用。
\begin{lstlisting}[language=Java, caption=Activity定义]
public class Activity{
    private Date date;//活动日期
    private Time time;//活动具体时间
    private String content;//活动内容
}
\end{lstlisting}

\subsubsection{\color{red} 数据结构9:SystemMessage\_server}
该数据结构定义了系统消息,继承自Message类。
\begin{lstlisting}[language=Java, caption=SystemMessage定义]
public class SystemMessage extends Message{
    private int type; //系统消息类型
}
\end{lstlisting}

\subsubsection{\color{red} 数据结构10:Error\_server}
该数据结构定义了报错的结构体。
\begin{lstlisting}[language=Java, caption=Error定义]
public class SystemMessage extends Message{
    private int position; //出错位置
    private int type; //错误类型
    private Date date; //出错日期
    private Time time; //出错时间
}
\end{lstlisting}


{\color{red}

\subsubsection{\color{red} 数据结构11:Survey\_server}
该数据结构定义了调查问卷的基本属性,以及问题的属性(包括选择题和问答题),将被用于信息调研模块。
\begin{lstlisting}[language=Java, caption=Survey定义]

public class Survey{
    private String title;//标题
    private String introduction;//简介
    private ArrayList<Question> questions;//问题列表
    private int mode;//模式
    private ArrayList<User>  
}


abstract class Question{
    String description;
}

class MultipleChoiceQuestion extends Question{
    String[] items;//选项
    ArrayList<User>[] choices; //选择每一个选项的用户列表
}
class AnswerQuestion extends Question{
    HashMap<User,String> answerMap;//每一个用户和他的答案
}
\end{lstlisting}

\subsubsection{\color{red} 数据结构12:Approval\_server}
该数据结构定义了审批流程的基本属性,用于审批模块。
\begin{lstlisting}[language=Java, caption=Approval定义]
public class Approval {
    private String content;//申请内容
    private File[] annex;//附件列表
    private int mode;//模式
    private ArrayList<User> managers;//有序的经办人列表
    private ArrayList<String> opinions;//每一个经办人的意见
    private ArrayList<Boolean> agreed;//每一个经办人是否批准
}
\end{lstlisting}

\subsubsection{\color{red} 数据结构13:Issue\_server}
该数据结构定义了Issue的基本属性,以及被其使用的回答类的属性。

\begin{lstlisting}[language=Java, caption=Issue定义]
public class Issue {
    private String question;//问题内容
    private User creater;//问题发布者
    private Time time;//发布时间
    private boolean closed;//问题是否关闭
    private ArrayList<Answer> answers;//回答列表
}
class Answer{
    private String content;//回答内容
    private User author;//回答作者
    private Time time;//发布时间
    private boolean recommended;//是否被issue发布者置顶
}
\end{lstlisting}
}

\subsection{\color{red}用户信息管理模块}
\begin{lstlisting}[language=Java, caption=用户信息管理模块接口]
public class UserInfoManager{
    public static int uploadUser(User user){
        // 将用户信息上传到服务器
        // 输入:User类实例,含有用户信息
        //     (User类定义见第5章数据结构设计)
        // 输出;状态码
    }
    public static User downloadUser(String userID){
        // 从服务器下载对应ID的用户信息
        // 输入:userID字符串,唯一标记操作对应的用户
        // 输出:下载的用户信息
    }
    public static User[] getRecommandedFriends(){
        // 向服务器获取个性化的推荐好友
        // 输入:无
        // 输出:推荐的好友用户信息列表
    }
    public static int sendFriendRequest(User user){
        // 发送好友请求
        // 输入:User类实例,含有好友请求对象的用户信息
        // 输出:状态码
    }
    public static SystemMessage[] receiveFriendRequest(){
        // 获取发送给自己的好友请求
        // 输入:无
        // 输出:从服务器接收的好友请求
        //(系统消息 SystemMessage类定义见第5章数据结构设计(Message子类))
    }
    public static int responseFriendRequest(SystemMessage request,boolean accept){
        // 对好友请求进行回复
        // 输入:
        //      request:回复的好友申请信息
        //      accept:该申请是否被通过
        // 输出: 状态码
    }
}

\end{lstlisting}




\subsection{\color{red}通讯管理模块}
{\color{red}

\begin{lstlisting}[language=Java, caption=通讯管理模块接口]
public class CommunicationManager{
    public static int sendMessage(Message message){
        // 发送信息
        // 输入:message:Message实例,需要发送的实例
        // 输出:状态码
    }
    public static Message receiveMessage(){
        // 接收信息(阻塞)
        // 输入:无
        // 输出:接收到的消息
    }
    public static boolean sendRadioConnectionRequest(User user, int type){
        // 发送音视频连接请求(阻塞+超时机制)
        // 输入:
        //      user:User实例,请求对象
        //      type:连接类型:0-音频连接,1-视频连接
        // 输出:连接是否成功
    }
    public static Connection receiveRadioConnectionRequest(){
        // 获取并回复发送给自己的音视频连接请求
        // 输入:无
        // 输出:接入的音视频连接
    }
    public static int sendRadio(Connection connection, byte[] data){
        // 发送音视频数据
        // 输入:
        //      connection:Connection实例,对应的音视频连接
        //      data:音视频数据流
        // 输出:状态码
    }
    public static byte[] receiveRadio(Connection connection){
        // 接收发送给自己的音视频数据
        // 输入:connection:Connection实例,对应的音视频连接
        // 输出:接收到的音视频数据流
    }
    public static void uploadMessage(Message message){
        // 上传聊天信息
        // 输入:message:Message实例,上传的信息
        // 输出:状态码
    }
    public static Message[] downloadMessage(User user, Date date){
        // 下载与指定对象在指定时间的聊天记录(null不进行指定)
        // 输入:
        //      user:User实例,查找聊天记录的对象
        //      date:Date实例,查找聊天记录对应的日期
        // 输出:得到的聊天记录列表
    }
    
    public static int publishIssue(Issue issue){
        // 发布一个issue
        // 输入:issue:Issue实例,发布的Issue信息
        // 输出:状态码
    }
    
    public static int answerIssue(Issue issue, Answer answer){
        // 回答指定的issue
        // 输入:
        //      issue:Issue实例,回答的Issue
        //      answer:Answer实例,回答的内容
        // 输出:状态码
    }
    
    public static int closeIssue(Issue issue){
        // 关闭指定的Issue
        // 输入:issue:Issue实例,关闭的Issue
        // 输出:状态码
    }
    
    
}
\end{lstlisting}
}


\subsection{\color{red}文件管理模块}
\begin{lstlisting}[language=Java, caption=文件管理模块接口]
public class FileManager{
    public static void saveFile(File file, String path){
        // 保存文件
        // 输入:
        //      file:File实例,需要保存的文件
        //      path:String实例,选定的保存文件的路径
        // 输出:状态码
    }
    public static File loadFile(String path){
        // 加载指定文件
        // 输入:path:String实例,需要加载的文件的路径
        // 输出:从服务器返回的文件
    }
    public static String[] searchFile(String name){
        // 查询指定文件
        // 输入:name:String实例,查找的文件名称
        // 输出:符合条件的文件名列表
    }
    public static Message sendFile(User user, File file){
        // 向指定用户发送文件,文件信息存放在返回的Message中。
        // 这个Message将被发送给对方,对方可以使用这个Message接收文件
        // 输入:
        //      user:User实例,发送对象的用户信息
        //      file:File实例,发送的文件
        // 输出:即将发送的Message
    }
    public static File receiveFile(Message message){
        // 使用存放有文件信息的Message接收文件
        // 输入:message:Message实例,接收到的含有文件的信息
        // 输出:从Message中取出的接收文件
    }

}
\end{lstlisting}

\subsection{\color{red}云文件管理模块}
\begin{lstlisting}[language=Java, caption=云文件管理模块接口]
public class CloudFileManager{
    public static String defaultURL;
    public static void uploadFile(File file, String url){
        // 上传云文件
        // 输入:
        //      file:上传的文件
        //      url:String实例,上传的文件的统一资源定位符
        //          (本地资源)
        // 输出:状态码
    }
    public static File downloadFile(String url){
        // 下载云文件
        // 输入:url:String实例,下载的文件的统一资源定位符
        // 输出:下载接收到的文件
    }
    public static String[] searchFile(String name){
        // 查询指定的云文件
        // 输入:name:String实例,查找的文件名
        // 输出:符合条件的云文件名称列表
    }
    public static String getFileInfo(String url){
        // 得到指定url对应的文件信息
        // 输入:url:String实例,查找的文件的统一资源定位符
        // 输出:查找的文件信息
    }
}
\end{lstlisting}

\subsection{\color{red}团队管理模块}
\begin{lstlisting}[language=Java, caption=团队管理模块接口]
public class GroupManager{
    public static void uploadFile(Group group, File file){
        // 上传群文件
        // 输入:
        //      group:Group实例,上传文件的群聊对象
        //      file:File实例,上传的文件
        // 输出:无
    }
    public static File downloadFile(Group group, String name){
        // 下载群文件
        // 输入:
        //      group:Group实例:
        //      name:String实例:
        // 输出:下载得到的文件
    }
    public static Group createGroup(User user, String name){
        // 创建群聊
        // 输入:
        //      user:
        //      name:
        // 输出:创建的新群组
    }
    public static Group getGroup(String groupID){
        // 得到对应ID的群聊
        // 输入:groupID
        // 输出:查找得到的群组
    }
    public static int disbandGroup(Group group){
        // 解散群聊
        // 输入:group
        // 输出:状态码
    }
    public static int updateGroupInfo(Group group, int type, User user){
        // 对群管理信息进行更新,包括增减群成员和改变群成员权限
        // 输入:Group: 群组
        //       type: 功能的类型,比如增加群成员,改变成员权限
        //       user: 被改变的成员
        //      
        // 输出:状态码
    }
    public static int publishTask(Group group, Task task){
        // 发布任务
        // 输入:group: 群组
        //      task:发布的任务
        //      
        // 输出:状态码
    }
    public static int updateTask(Group group, Task task){
        // 更新任务
        // 输入:group:群组
        //      task:更新的任务
        //      
        // 输出:状态码
    }
}
\end{lstlisting}

\subsection{\color{red}日历管理模块}
\begin{lstlisting}[language=Java, caption=日历管理模块接口]
public class CalanderManager{
    public static Activity[][] calanderBuffer;
    public static Activity[] downloadActivities(){
        // 下载活动项
        // 输入:无
        // 输出:Activity[]:日程信息列表
    }
    public static int uploadActivity(Activity activity){
        // 上传活动项
        // 输入:Activity:日程信息
        // 输出:状态码
    }
    public static int deleteActivity(Activity activity){
        // 删除活动项
        // 输入:Activity:日程信息
        // 输出:状态码
    }
    public static int addActivity(Activity activity){
        // 新增活动项
        // 输入:Activity: 日程信息
        // 输出:状态码
    }
    public static int getActivities(Date date,int length){
        // 获取某一段时间的所有活动,写入calanderBuffer
        // 输入:Date: 日期,length: 活动的数目
        // 输出:状态码
    }
}
\end{lstlisting}

\subsection{\color{red}共享文档管理模块}
\begin{lstlisting}[language=Java, caption=共享文档管理模块接口]
public class DocumentCooperationManager{
    public static String createDocument(File file){
        // 创建共享文档,返回URL
        // 输入:file,File实例,初始化的文件
        // 输出:创建的共享文档的连接URL
    }
    public static File openDocument(String url){
        // 使用URL打开共享文档
        // 输入:url:String实例,需要打开的共享文档的统一资源定位符
        // 输出:打开的文件,本地或从服务器获得的副本
    }
    public static int sendUpdate(String url,Update update){
        // 发送更改
        // 输入: 
        //      update:更新信息
        //      url:URL
        // 输出:状态码
    }
    public static Update receiveUpdate(){
        // 接收更改
        // 输入:无
        // 输出:从服务器接收到的更改信息
    }
    public static File applyUpdate(File file,Update update){
        // 应用更改
        // 输入:
        //      file:共享文档
        //      update:更新的内容
        // 输出:修改之后的文件
    }
}
\end{lstlisting}

\subsection{\color{red}错误管理模块}
\begin{lstlisting}[language=Java, caption=错误管理模块接口]
public class ErrorManager{
    public static void reportError(Exception ex){
        // 向服务器报告错误
        // 输入:ex:Exception实例,报告的错误
        // 输出:状态码
    }
}
\end{lstlisting}


{\color{red}

\subsection{\color{red} 第三方信息获取模块}
\begin{lstlisting}[language=Java, caption=爬虫模块接口]
public class CrawlerManager{
    public static String fetch(URL url){
        // 爬取网页,并返回重要信息
        // 输入:url:URL实例,需要爬取的网页的地址
        // 输出:网页内容
    } 
    public static String[] analyse(String s){
        // 解析网页信息
        // 输入:s:String实例,爬取到的网页
        // 输出:网页元素列表
    }
}
\end{lstlisting}

\subsection{\color{red}信息调研模块}
\begin{lstlisting}[language=Java, caption=信息调研模块接口]
public class SurveyManager{
    public static Survey downloadSurvey(){
        // 下载问卷
        // 输入:无
        // 输出:下载得到的问卷
    }
    public static int uploadSurvey(Survey survey){
        // 上传问卷
        // 输入:survey:Survey实例,上传的问卷
        // 输出:状态码
    }
    public static int updateSurvey(Survey survey){
        // 更新问卷
        // 输入:survey:Survey实例,更新的问卷
        // 输出:状态码
    }
}
\end{lstlisting}


\subsection{\color{red}流程审批模块}
\begin{lstlisting}[language=Java, caption=流程模块接口]
public class ApprovalManager{
    public static Approval downloadApproval(){
        // 下载审批
        // 输入:无
        // 输出:下载得到的审批
    }
    public static void uploadApproval(Approval approval){
        // 上传审批
        // 输入:approval:Approval实例,上传的审批
        // 输出:状态码
    }
    public static void acceptApproval(Approval approval,String opinion){
        // 批准审批
        // 输入:
        //      approval:Approval实例,批准的审批
        //      opinion:String实例,批准审批意见
        // 输出:状态码
    }
    public static void refuseApproval(Approval approval,String opinion){
        // 不批准审批
        // 输入:
        //      approval:Approval实例,被拒的审批
        //      opinion:String实例,拒绝审批意见
        // 输出:状态码
    }
}
\end{lstlisting}


}
