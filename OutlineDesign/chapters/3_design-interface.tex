\chapter{接口设计}
%====================================================================================
\section{外部接口}
%====================================================================================
% 比如说需要用到支付宝等外部支付系统,接口应当如何封装。
%====================================================================================
JavaMail是由Sun发布的用来处理email的API,它是Java EE的一部分。我们调用该接口查询和发送邮件。

\section{内部接口}
\subsection{用户信息管理模块}
\begin{lstlisting}[language=Java, caption=用户信息管理模块接口]
public class UserInfoManager{
    public static void uploadUser(User user){
        //将用户信息上传到服务器
    }
    public static User downloadUser(String userID){
        //从服务器下载对应ID的用户信息
    }
    public static User[] getRecommandedFriends(){
        //向服务器获取个性化的推荐好友
    }
    public static void sendFriendRequest(User user){
        //发送好友请求
    }
    public static SystemMessage[] receiveFriendRequest(){
        //获取发送给自己的好友请求
    }
    public static void responseFriendRequest(
        SystemMessage request,boolean accept){
        //对好友请求进行回复
    }
}

\end{lstlisting}




\subsection{通讯管理模块}
{\color{red}

\begin{lstlisting}[language=Java, caption=通讯管理模块接口]
public class CommunicationManager{
    public static void sendMessage(Message message){
        //发送信息
    }
    public static Message receiveMessage(){
        //接收信息(阻塞)
    }
    public static boolean sendRadioConnectionRequest(
        User user,int type){
        //发送音视频连接请求(阻塞+超时机制)
    }
    public static Connection 
        receiveRadioConnectionRequest(){
        //获取并回复发送给自己的音视频连接请求
    }
    public static void sendRadio(Connection connection,
        byte[] data){
        //发送音视频数据
    }
    public static byte[] receiveRadio(
        Connection connection){
        //接收发送给自己的音视频数据
    }
    public static void uploadMessage(Message message){
        //上传聊天信息
    }
    public static Message[] downloadMessage(
        User user,Date date){
        //下载与指定对象在指定时间的聊天记录(null不进行指定)
    }
    
    public static void publishIssue(Issue issue){
        //发布一个issue
    }
    
    public static void answerIssue(Issue issue, Answer answer){
        //回答指定的issue
    }
    
    public static void closeIssue(Issue issue){
        //关闭指定的Issue
    }
    
    
}
\end{lstlisting}
}
\subsection{文件管理模块}
\begin{lstlisting}[language=Java, caption=文件管理模块接口]
public class FileManager{
    public static void saveFile(File file,String path){
        //保存文件
    }
    public static File loadFile(String path){
        //加载指定文件
    }
    public static String[] searchFile(String name){
        //查询指定文件
    }
    public static Message sendFile(User user,File file){
        //向指定用户发送文件,文件信息存放在返回的Message中。
        //这个Message将被发送给对方,对方可以使用这个Message接收文件
    }
    public static File receiveFile(Message message){
        //使用存放有文件信息的Message接收文件
    }

}
\end{lstlisting}

\subsection{云文件管理模块}
\begin{lstlisting}[language=Java, caption=云文件管理模块接口]
public class CloudFileManager{
    public static String defaultURL;
    public static void uploadFile(File file,String url){
        //上传云文件
    }
    public static File downloadFile(String url){
        //下载云文件
    }
    public static String[] searchFile(String name){
        //查询指定的云文件
    }
    public static String getFileInfo(String url){
        //得到指定url对应的文件信息
    }
}
\end{lstlisting}

\subsection{团队管理模块}
\begin{lstlisting}[language=Java, caption=团队管理模块接口]
public class GroupManager{
    public static void uploadFile(Group group,File file){
        //上传群文件
    }
    public static File downloadFile(Group group,
        String name){
        //下载群文件
    }
    public static Group createGroup(User user,String name){
        //创建群聊
    }
    public static Group getGroup(String groupID){
        //得到对应ID的群聊
    }
    public static void disbandGroup(Group group){
        //解散群聊
    }
    public static void updateGroupInfo(Group group,
        int type,User user){
        //对群管理信息进行更新,包括增减群成员和改变群成员权限
    }
    public static void publishTask(Group group,Task task){
        //发布任务
    }
    public static void updateTask(Group group,Task task){
        //更新任务
    }
}
\end{lstlisting}

\subsection{日历管理模块}
\begin{lstlisting}[language=Java, caption=日历管理模块接口]
public class CalanderManager{
    public static Activity[][] calanderBuffer;
    public static Activity[] downloadActivities(){
        //下载活动项
    }
    public static void uploadActivity(Activity activity){
        //上传活动项
    }
    public static void deleteActivity(Activity activity){
        //删除活动项
    }
    public static void addActivity(Activity activity){
        //新增活动项
    }
    public static void getActivities(Date date,int length){
        //获取某一段时间的所有活动,写入calanderBuffer
    }
}
\end{lstlisting}

\subsection{共享文档管理模块}
\begin{lstlisting}[language=Java, caption=共享文档管理模块接口]
public class DocumentCooperationManager{
    public static String createDocument(File file){
        //创建共享文档,返回URL
    }
    public static File openDocument(String url){
        //使用URL打开共享文档
    }
    public static void sendUpdate(String url,Update update){
        //发送更改
    }
    public static Update receiveUpdate(){
        //接收更改
    }
    public static File applyUpdate(File file,Update update){
        //应用更改
    }
}
\end{lstlisting}

\subsection{错误管理模块}
\begin{lstlisting}[language=Java, caption=错误管理模块接口]
public class ErrorManager{
    public static void reportError(Exception ex){
        //向服务器报告错误
    }
}
\end{lstlisting}


{\color{red}

\subsection{\color{red}爬虫模块}
\begin{lstlisting}[language=Java, caption=爬虫模块接口]
public class CrawlerManager{
    public static String fetch(URL url){
        //爬取网页,并返回重要信息
    } 
    public static String[] analyse(String s){
        //解析网页信息
    }
}
\end{lstlisting}

\subsection{\color{red}信息调研模块}
\begin{lstlisting}[language=Java, caption=信息调研模块接口]
public class SurveyManager{
    public static Survey downloadSurvey(){
        //下载问卷
    }
    public static void uploadSurvey(Survey survey){
        //上传问卷
    }
    public static void updateSurvey(Survey survey){
        //更新问卷
    }
}
\end{lstlisting}


\subsection{\color{red}流程审批模块}
\begin{lstlisting}[language=Java, caption=流程模块接口]
public class ApprovalManager{
    public static Approval downloadApproval(){
        //下载审批
    }
    public static void uploadApproval(Approval approval){
        //上传审批
    }
    public static void acceptApproval(Approval approval,String opinion){
        //批准审批
    }
    public static void refuseApproval(Approval approval,String opinion){
        //不批准审批
    }
}
\end{lstlisting}


}
