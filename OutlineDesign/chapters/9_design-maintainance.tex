\chapter{\color{red} 维护设计}
可能的内容包括数据库的日常备份、压缩、维护等。
\section{\color{red} 日常备份}
1、每周对数据库进行一次完全备份。
2、每天夜里2:00 am - 3:00 am 对数据库的事务日志进行差异备份。
3、每天保留最近两天的数据库和事务日志的备份,自动地删除久于两天前的所有数据库和事务日志的备份。
4. 对通讯服务及通讯数据进行同步热备份

除日志以外,需要完全备份的数据库数据为:
{\color{red}
\begin{itemize}
    \item 用户数据表
    \item 团队数据表
    \item 日历数据表
    \item 文件数据表
    \item issue数据表
    \item 协作文档数据表
    \item 包含关系数据表
    \item 发布关系数据表
    \item 管理关系数据表
    \item 编辑关系数据表
    \item 好友关系数据表
    \item 聊天关系数据表
    \item 收发关系数据表
    {\color{red}
    \item 调研关系数据表
    \item 审批关系数据表
    }
    
    
\end{itemize}
}
\section{数据压缩}

为节约空间同时保证系统性能,压缩处理如下:
\begin{itemize}
    \item 对于时间超过1年且未访问的数据进行数据压缩,例如聊天记录,文件
    \item 对多媒体数据如音视频等进行压缩存储
    \item 对于低级备份文件(磁盘备份)进行压缩存储
\end{itemize}


\section{数据库安全}

保障数据库安全采用以下方法:
\begin{itemize}
    \item 1. 用户标识与鉴别: 使用安全的密码策略, 使用安全的账号策略
    \item 2. 对用户定义视图,对查询进行限制
    \item 3. 数据加密:使用协议加密
    \item 4. 安全审计:对网络连接进行访问IP限制,加强数据库日志的记录
\end{itemize}

实现数据库安全的框架为:
\begin{itemize}
    \item 事前诊断\\
    1)分析内部不安全配置,防止越权访问:通过只读账户,实现由内到外的检测;提供现有数据的漏洞透视图和数据库配置安全评估;避免内外部的非授权访问。\\
    2)监控数据库安全状况,防止数据库安全状况恶化:对于数据库建立安全基线,对数据库进行定期扫描,对所有安全状况发生的变化进行报告和分析。
    \item 事中控制 \\
    对数据库中的敏感数据加密存储、访问控制增强、应用访问安全、安全审计以及三权分立等功能。
    \item 事后分析 \\
    对于安全事件,对数据库系统进行更新,补丁,从而完成对主要数据库漏洞的防控。
    
\end{itemize}


\section{数据库维护}

日常监视系统运行状况\\

监视系统运行情况,及时处理系统错误。主要有以下几个方面:
\begin{itemize}
    

\item 1、监视当前用户以及进程的信息
\item 2、监视目标占用空间情况
\item 3. 监视网络的连接情况:对于即时通讯系统至关重要,包括带宽,速度等
\item 4. 检查数据库系统中数据的机密性,正确性,可用性

\end{itemize}

