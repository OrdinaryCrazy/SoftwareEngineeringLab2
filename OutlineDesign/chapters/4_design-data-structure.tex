\chapter{\color{red} 数据结构设计}
%=================================================================================
{\color{red}
    \section{{\color{red}逻辑结构设计}}
    %-----------------------------------------------------------------------------

        %================================ GUIDE ==================================
        % 讲述本系统内需要什么数据结构。这指的是程序运行过程中维护的数据结构。
        % 只是举个例子,此处应和3.3一致。
        %================================ GUIDE ==================================
    %-----------------------------------------------------------------------------
        \subsection{{\color{red}客户端数据结构}}
}        
    %-----------------------------------------------------------------------------

        
    %-----------------------------------------------------------------------------
\subsubsection{数据结构1:User\_client}
该数据结构定义了用户的基本属性。
\begin{lstlisting}[language=Java, caption=User定义]
public class User{
    private String ID;//用户ID,唯一
    private String name;//昵称,不唯一
    private String sex="Male";//性别,默认男性
    private Date birthday;//用户的出生日期,可以为空
    private Time createTime;//账户创建时间
    private String address;//用户地址,可以为空
    private String email;//用户绑定的邮箱,可以为空
    public User(String ID){

    }
    public User(String ID,String name,String sex,
        Date birthday,String address,String email){

    }
    public String getID(){

    }
    public String getName(){

    }
    public void setName(String name){

    }
    public String getSex(){

    }
    public void setSex(String sex){

    }
    public Date getBirthday(){

    }
    public void setBirthday(Date birthday){

    }
    public int getAge(){

    }
    public Time getCreateTime(){

    }
    public void setCreateTime(Time createTime){

    }
    public String getAddress(){

    }
    public void setAddress(String address){

    }
    public String getEmail(){

    }
    public void setEmail(String email){

    }
}
\end{lstlisting}

\subsubsection{数据结构2:Message\_client}
该数据结构定义了一对一聊天时发送的信息,所有属性都是不可修改的。此外,在发送文件时,该数据结构也会被使用,用于声明文件的一些基本信息(使用message域)。
\begin{lstlisting}[language=Java, caption=Message定义]
public class Message{
    private User sender,receiver;//该信息的发送方和接收方
    private String message;//信息内容
    private Time sendtime;//发送时间
    public Message(User sender,User receiver,String message,
        Time sendtime){

    }
    public User getSender(){

    }
    public User getReceiver(){

    }
    public String getMessage(){
        
    }
    public Time getSendtime(){

    }
}
\end{lstlisting}
    
\subsubsection{数据结构3:GroupMessage\_client}
该数据结构定义了群聊时发送的信息,继承自Message类。同样,所有属性都是不可修改的。
\begin{lstlisting}[language=Java, caption=GroupMessage定义]
public class GroupMessage extends Message{
    private Group group;//所属的群聊
    public GroupMessage(Group group,User sender,
        User receiver, String message,Time sendtime){

    }
    public Group getGroup(){

    }
}
\end{lstlisting}

\subsubsection{数据结构4:Group\_client}
该数据结构定义了群聊的基本属性。
\begin{lstlisting}[language=Java, caption=Group定义]
public class Group{
    private String ID;//群ID,唯一
    private String name;//群名称,不唯一
    private User master;//群主
    private User[] managerList;//管理员列表
    private User[] memberList;//群成员列表
    private Scene scene; //场景
    public Group(String ID,User master){

    }
    public String getID(){

    }
    public String getName(){

    }
    public void setName(String name){

    }
    public User getMaster(){

    }
    public User[] getManagerList(){

    }
    public User[] getMemberList(){

    }
    public void appointManager(User user){
    //任命管理员
    }
    public void dismissManager(User user){
    //解除管理员职务
    }
    public void addMember(User user){
    //增加群成员
    }
    public void deleteMember(User user){
    //删除群成员
    }
    public Scene getScene(){

    }
    public void setScene(Scene scene){
        
    }
}
\end{lstlisting}

\subsubsection{数据结构5:Friend\_client}
该数据结构定义了通讯录中好友的基本属性。通讯录可以对好友分组管理,这是通过tag实现的,具有相同tag的用户被认为是一个组。
\begin{lstlisting}[language=Java, caption=Friend定义]
public class Friend{
    private User user;//用户
    private String tag;//分组标记,内容是组名
    private String remark;//备注名
    public Friend(User user,String tag,String remark){

    }
    public User getUser(){

    }
    public String getTag(){

    }
    public void setTag(String tag){

    }
    public String getRemark(){

    }
    public void setRemark(String remark){

    }
}
\end{lstlisting}

\subsubsection{数据结构6:AddressBook\_client}
该数据结构定义了通讯录,包括好友列表,群聊列表和黑名单。
\begin{lstlisting}[language=Java, caption=AddressBook定义]
public class AddressBook{
    private Friend[] friendList;//好友列表
    private User[] blackList;//黑名单
    private Group[] groupList;//群聊列表
    public AddressBook(){

    }
    public Friend[] getFriendList(){

    }
    public User[] getBlackList(){

    }
    public Group[] getGroupList(){

    }
    public void addFriend(User user){

    }
    public void addFriend(Friend friend){

    }
    public void deleteFriend(Friend friend){
        
    }
    public void addBlack(User user){

    }
    public void deleteBlack(User user){

    }
    public void addGroup(Group group){

    }
    public void deleteGroup(Group group){

    }
    public Friend[] findFriendByTag(String tag){
        //查找某一组的所有好友
    }
}
\end{lstlisting}

\subsubsection{数据结构7:Task\_client}
该数据结构定义了任务的基本属性,它将在活动/任务发布与管理功能中应用。
\begin{lstlisting}[language=Java, caption=Task定义]
public class Task{
    private User organizer;//组织者
    private User[] participants;//参与者列表
    private String address;//任务地点
    private Time time;//任务时间
    private String content;//任务内容
    public Task(){

    }
    public User getOrganizer(){

    }
    public User[] getParticipants(){

    }
    public void setParticipants(User[] participants){

    }
    public void addParticipant(User participant){

    }
    public void deleteParticipant(User participant){

    }
    public String getAddress(){

    }
    public void setAddress(String address){

    }
    public Time getTime(){

    }
    public void setTime(Time time){

    }
    public String getContent(){

    }
    public void setContent(){

    }
}
\end{lstlisting}

\subsubsection{数据结构8:Activity\_client}
该数据结构定义了活动项的基本属性,将被用于日历应用。
\begin{lstlisting}[language=Java, caption=Activity定义]
public class Activity{
    private Date date;//活动日期
    private Time time;//活动具体时间
    private String content;//活动内容
    public Activity(Date date, Time time, String content){

    }
    public Date getDate(){

    }
    public void setDate(Date date){

    }
    public Time getTime(){

    }
    public void setTime(){

    }
    public String getContent(){

    }
    public void setContent(String content){
        
    }
}
\end{lstlisting}

\subsubsection{数据结构9:SystemMessage\_client}
该数据结构定义了系统消息,继承自Message类。
\begin{lstlisting}[language=Java, caption=SystemMessage定义]
public class SystemMessage extends Message{
    private int type; //系统消息类型
    public SystemMessage(User sender,User reciver,
        String message,Time sendtime,int type){

    } 
    public int setType(){

    }
}
\end{lstlisting}

\subsubsection{数据结构10:Update\_client}
该数据结构定义了对共享文档的原子更改,所有属性都是不可更改的。
\begin{lstlisting}[language=Java, caption=Update定义]
public class Update{
    private int pos;//修改位置
    private int type;//修改类型(增、删、改)
    private int length;//修改长度
    private String updateText;//修改后的内容
    private String url;//修改的文件的URL
    public Update(int pos,int type,int length,
        String updateText,String url){

    }
    public int getPos(){

    }
    public int getType(){

    }
    public int getLength(){

    }
    public Stirng getUpdateText(){

    }
    public String getURL(){

    }
}
\end{lstlisting}

{\color{red}

\subsubsection{数据结构11:Survey\_client}
该数据结构定义了调查问卷的基本属性,以及问题的属性(包括选择题和问答题),将被用于信息调研模块。
\begin{lstlisting}[language=Java, caption=Survey定义]

public class Survey{
    private String title;//标题
    private String introduction;//简介
    private ArrayList<Question> questions;//问题列表
    private int mode;//模式
    private ArrayList<User> respondents;//允许填写问卷的用户列表,null代表所有人

    public String getTitle() {
        return title;
    }
    public void setTitle(String title) {
        this.title = title;
    }
    public String getIntroduction() {
        return introduction;
    }
    public void setIntroduction(String introduction) {
        this.introduction = introduction;
    }
    public ArrayList<Question> getQuestions() {
        return questions;
    }
    public void setQuestions(ArrayList<Question> questions) {
        this.questions = questions;
    }

    public int getMode() {
        return mode;
    }

    public void setMode(int mode) {
        this.mode = mode;
    }

    public ArrayList<User> getRespondents() {
        return respondents;
    }

    public void setRespondents(ArrayList<User> respondents) {
        this.respondents = respondents;
    }   
}


abstract class Question{
    String description;
}

class MultipleChoiceQuestion extends Question{
    String[] items;//选项
    ArrayList<User>[] choices; //选择每一个选项的用户列表

    public MultipleChoiceQuestion(String[] items) {

    }
    public void insertChoice(User user,int item){
        
    }   
}
class AnswerQuestion extends Question{
    HashMap<User,String> answerMap;//每一个用户和他的答案

    public AnswerQuestion() {
    }
    public void insertAnswer(User user,String answer){
        
    }
}
\end{lstlisting}

\subsubsection{数据结构12:Approval\_client}
该数据结构定义了审批流程的基本属性,用于审批模块。
\begin{lstlisting}[language=Java, caption=Approval定义]
public class Approval {
    private String content;//申请内容
    private File[] annex;//附件列表
    private int mode;//模式
    private ArrayList<User> managers;//有序的经办人列表
    private ArrayList<String> opinions;//每一个经办人的意见
    private ArrayList<Boolean> agreed;//每一个经办人是否批准

    public String getContent() {
        return content;
    }

    public void setContent(String content) {
        this.content = content;
    }

    public File[] getAnnex() {
        return annex;
    }

    public void setAnnex(File[] annex) {
        this.annex = annex;
    }

    public int getMode() {
        return mode;
    }

    public void setMode(int mode) {
        this.mode = mode;
    }

    public ArrayList<User> getManagers() {
        return managers;
    }

    public void setManagers(ArrayList<User> managers) {
        this.managers = managers;
    }

    public ArrayList<String> getOpinions() {
        return opinions;
    }

    public void setOpinions(ArrayList<String> opinions) {
        this.opinions = opinions;
    }

    public ArrayList<Boolean> getAgreed() {
        return agreed;
    }

    public void setAgreed(ArrayList<Boolean> agreed) {
        this.agreed = agreed;
    }
    
}
\end{lstlisting}

\subsubsection{数据结构13:Issue\_client}
该数据结构定义了Issue的基本属性,以及被其使用的回答类的属性。

\begin{lstlisting}[language=Java, caption=Issue定义]
public class Issue {
    private String question;//问题内容
    private User creater;//问题发布者
    private Time time;//发布时间
    private boolean closed;//问题是否关闭
    private ArrayList<Answer> answers;//回答列表

    public Issue() {
    }
    
    public String getQuestion() {
        return question;
    }

    public void setQuestion(String question) {
        this.question = question;
    }

    public User getCreater() {
        return creater;
    }

    public void setCreater(User creater) {
        this.creater = creater;
    }

    public boolean isClosed() {
        return closed;
    }

    public void setClosed(boolean closed) {
        this.closed = closed;
    }

    public ArrayList<Answer> getAnswers() {
        return answers;
    }

    public void setAnswers(ArrayList<Answer> answers) {
        this.answers = answers;
    }

    public Time getTime() {
        return time;
    }

    public void setTime(Time time) {
        this.time = time;
    }
    
}

class Answer{
    private String content;//回答内容
    private User author;//回答作者
    private Time time;//发布时间
    private boolean recommended;//是否被issue发布者置顶

    public Answer(String content, User author, Time time) {

    }
    
    public String getContent(){
    
    }
    
    public void setContent(String content){
    
    }
    
    public boolean getRecommend(){
        
    }
    public void setRecommend(boolean b){
        
    }
}
\end{lstlisting}


}


{\color{red}



\subsection{{\color{red}服务器端数据结构}}
}
%--------
\subsubsection{\color{red} 数据结构1:User\_server}
该数据结构定义了用户的基本属性。
\begin{lstlisting}[language=Java, caption=User定义]
public class User{
    private String ID;//用户ID,唯一
    private String name;//昵称,不唯一
    private String sex="Male";//性别,默认男性
    private Date birthday;//用户的出生日期,可以为空
    private Time createTime;//账户创建时间
    private String address;//用户地址,可以为空
    private String email;//用户绑定的邮箱,可以为空
    private ArrayList<String> tpIDs; //用户的第三方平台账号
    private ArrayList<String> tpPasswords; //用户的第三方平台密码
}
\end{lstlisting}

\subsubsection{数据结构2:Message\_server}
该数据结构定义了一对一聊天时发送的信息,所有属性都是不可修改的。此外,在发送文件时,该数据结构也会被使用,用于声明文件的一些基本信息(使用message域)。
\begin{lstlisting}[language=Java, caption=Message定义]
public class Message{
    private User sender,receiver;//该信息的发送方和接收方
    private String message;//信息内容
    private Time sendtime;//发送时间
}
\end{lstlisting}

\subsubsection{数据结构3:GroupMessage\_server}
该数据结构定义了群聊时发送的信息,继承自Message类。同样,所有属性都是不可修改的。
\begin{lstlisting}[language=Java, caption=GroupMessage定义]
public class GroupMessage extends Message{
    private Group group;//所属的群聊
}
\end{lstlisting}

{\color{red}
\subsubsection{数据结构4:Group\_server}
该数据结构定义了群聊的基本属性。
\begin{lstlisting}[language=Java, caption=Group定义]
public class Group{
    private String ID;//群ID,唯一
    private String name;//群名称,不唯一
    private User master;//群主
    private User[] managerList;//管理员列表
    private User[] memberList;//群成员列表
    private Scene scene; //场景
    private ArrayList<Issue_server> issuelist;//本群中所有的issue
    
\end{lstlisting}
}


\subsubsection{数据结构5:Friend\_server}
该数据结构定义了通讯录中好友的基本属性。通讯录可以对好友分组管理,这是通过tag实现的,具有相同tag的用户被认为是一个组。
\begin{lstlisting}[language=Java, caption=Friend定义]
public class Friend{
    private User user;//用户
    private String tag;//分组标记,内容是组名
    private String remark;//备注名
}
\end{lstlisting}


\subsubsection{数据结构6:AddressBook\_server}
该数据结构定义了通讯录,包括好友列表,群聊列表和黑名单。
\begin{lstlisting}[language=Java, caption=AddressBook定义]
public class AddressBook{
    private Friend[] friendList;//好友列表
    private User[] blackList;//黑名单
    private Group[] groupList;//群聊列表
}
\end{lstlisting}

\subsubsection{数据结构7:Task\_server}
该数据结构定义了任务的基本属性,它将在活动/任务发布与管理功能中应用。
\begin{lstlisting}[language=Java, caption=Task定义]
public class Task{
    private User organizer;//组织者
    private User[] participants;//参与者列表
    private String address;//任务地点
    private Time time;//任务时间
    private String content;//任务内容
}
\end{lstlisting}

\subsubsection{数据结构8:Activity\_server}
该数据结构定义了活动项的基本属性,将被用于日历应用。
\begin{lstlisting}[language=Java, caption=Activity定义]
public class Activity{
    private Date date;//活动日期
    private Time time;//活动具体时间
    private String content;//活动内容
}
\end{lstlisting}

\subsubsection{数据结构9:SystemMessage\_server}
该数据结构定义了系统消息,继承自Message类。
\begin{lstlisting}[language=Java, caption=SystemMessage定义]
public class SystemMessage extends Message{
    private int type; //系统消息类型
}
\end{lstlisting}

\subsubsection{数据结构10:Error\_server}
该数据结构定义了报错的结构体。
\begin{lstlisting}[language=Java, caption=Error定义]
public class SystemMessage extends Message{
    private int position; //出错位置
    private int type; //错误类型
    private Date date; //出错日期
    private Time time; //出错时间
}
\end{lstlisting}


{\color{red}

\subsubsection{数据结构11:Survey\_server}
该数据结构定义了调查问卷的基本属性,以及问题的属性(包括选择题和问答题),将被用于信息调研模块。
\begin{lstlisting}[language=Java, caption=Survey定义]

public class Survey{
    private String title;//标题
    private String introduction;//简介
    private ArrayList<Question> questions;//问题列表
    private int mode;//模式
    private ArrayList<User>  
}


abstract class Question{
    String description;
}

class MultipleChoiceQuestion extends Question{
    String[] items;//选项
    ArrayList<User>[] choices; //选择每一个选项的用户列表
}
class AnswerQuestion extends Question{
    HashMap<User,String> answerMap;//每一个用户和他的答案
}
\end{lstlisting}

\subsubsection{数据结构12:Approval\_server}
该数据结构定义了审批流程的基本属性,用于审批模块。
\begin{lstlisting}[language=Java, caption=Approval定义]
public class Approval {
    private String content;//申请内容
    private File[] annex;//附件列表
    private int mode;//模式
    private ArrayList<User> managers;//有序的经办人列表
    private ArrayList<String> opinions;//每一个经办人的意见
    private ArrayList<Boolean> agreed;//每一个经办人是否批准
}
\end{lstlisting}

\subsubsection{数据结构13:Issue\_server}
该数据结构定义了Issue的基本属性,以及被其使用的回答类的属性。

\begin{lstlisting}[language=Java, caption=Issue定义]
public class Issue {
    private String question;//问题内容
    private User creater;//问题发布者
    private Time time;//发布时间
    private boolean closed;//问题是否关闭
    private ArrayList<Answer> answers;//回答列表
}
class Answer{
    private String content;//回答内容
    private User author;//回答作者
    private Time time;//发布时间
    private boolean recommended;//是否被issue发布者置顶
}
\end{lstlisting}
}




\subsubsection{数据结构14:UserInfoManagement\_server}

\begin{lstlisting}[language=Java, caption=用户信息管理类]
public class UserInfoManagement{
    private static User[] Users;//存储所有的用户
    private static User[] OnlineUsers; //存储所有的在线用户
    
    public static void registerUser(User_server user){
        //用户注册
    }
    public static int selectUser(User_server user){
        //在Users列表中查询用户对应项
    }
    public static User_server[] recommendUser(User_server
    user){
        //推荐好友列表
    }
    public static AddressBook(
    
 User_server user){
        //查找某个user的联系人列表
    }   pserver(public static void deleteUser(User_server 
    user){
        //删除用户
    }
    public static void UpdateUser(int UserID){
        //更新Users列表中的User信息
        //(例如绑定邮箱后需更新Users列表中对应user的email项)
    }
    public static void upLoad(User_server user){
        //向数据库上传
    }
    public static User[] downLoad(User_server user){
        //从数据库下载用户信息,并打包为User结构体
    }
\end{lstlisting}

%--------
\subsubsection{数据结构15:CommunicationManagement\_server}
%--------
\begin{lstlisting}[language=Java, caption=通讯管理类]
public class CommunicationManagement{
    private static Message[] messageBuffer; 
    //存储待转发的消息队列
    
    public static Message_server Receive(Message_server
    
    message_received){
        //接收消息
    }
    public static void Send(Message_server message_send){
        //发送消息
    }
    public static void Transfer(User_server sender, 
    User\_server Receiver){
        //转发消息
    )
    public static void upLoad(Message_server message){
        //向数据库上传聊天记录
    }
    public static Message_server downLoad(Message_server 
    message){
        //从数据库下载聊天记录,并打包为Message类
    }
}
\end{lstlisting}
%--------
\subsubsection{数据结构16:FileManagement\_server}
%--------
\begin{lstlisting}[language=Java, caption=文件管理类]
public class FileManagement{
    private static Message_server[]  FileRecord; 
    //记录所有的文件信息及其存储地址
    
    public static int selectRecord(Message_server record){
        //查询文件记录在FileRecord中的对应项
    }
    
    public static Message_server ReceiveFileRecord(){
        //接收文件记录(其为Message的一种类型,记录了发送者,
        接收者,存储地址等信息)
    }
    public static File ReceiveFile(int FileID){
        //接收文件, 并将其绑定到对应的文件记录
    }
    public static void SendFileRecord(Message_server 
    filerecord){
        //发送文件记录(其为Message的一种类型,记录了发送者,
        接收者,存储地址等信息)
    }
    public static void SendFile(Message_server filerecord,
    File file){
        //发送文件
    }
    public static void Transfer(Message_server filerecord,
    File file){
        //转发文件
    }
    public static void upLoad(Message_server record, 
    File file){
        //向数据库上传文件
    }
    public static File downLoad(Message_server record){
        //根据文件记录中的地址,从数据库下载文件
    }
    
}
\end{lstlisting}
%--------
\subsubsection{数据结构17:CalendarManagement\_server}
%--------
\begin{lstlisting}[language=Java, caption=日历管理类]
public class CalendarManagement{
    public static void Activity_server[][] calendarBuffer;
    
    //存储需要修改的日历
    
    public static int[] selectCalendar(Activity_server
    activity){
        //查找日历缓存表中的对应日历项
    }
    public static void addActivity(int[] calendarID, 
    Activity_server activity){
        //向单个或多个日历中添加活动
    }
    public static void deleteActivity(int[] calendarID, 
    Activity_server activity){
        //向单个或多个日历中删除活动
    }
    public static void updateActivity(int [] calendarID, 
    Activity_server activity){
        //向单个或多个日历中更新活动
    }
    public static void upLoad(Activity_server
    [] Calendar){
        //向数据库上传日历
    }
    public static Activity_server[] downLoad(
    Activity_server[] Calendar){
        //从数据库下载日历
    }
    
}
\end{lstlisting}
%--------
{\color{red}
\subsubsection{\color{red} 数据结构18:GroupManagement\_server}
%--------
\begin{lstlisting}[language=Java, caption=团队管理类]
public static class GroupManagement{
    public static Group_server[] Groups;
    
    public static int selectGroup(Group_server group){
        //在群组管理列表中寻找对应项
    }
    public static void Assign_Task(Task_server[] task){
        //向组员发布任务
    }
    public static void setAuthority(Group_server group){
        //设置群组中用户的权限
    }
    public static void reflect(SystemMessage_server 
    message){
        //根据组员的反馈信息调整任务
    }
    public static void upLoad(Group_server[] group){
        //向数据库上传团队信息
    }
    public static Group_server[] downLoad(Group_server[] 
    group){
        //从数据库下载团队信息
    }
    
    public static Issue_server[] GetIncompletedIssues(){
        //获取未解决的issue
    }
    public static Issue_server[] GetCompletedIssues(){
        //获取已经解决的issue
    }
    public static void upLoadIssue(Issue_server[]){
        //向issue记录存储数据库上传文件
    }
    public static Issue_server[] downLoadIssue(){
        //从issue记录存储数据库下载文件
    }
    public static Answer recvSolution(){
        //从团队用户处获取issue的解决方案
    }
    public static void sendQuestion(){
        //向团队用户发送未解决的issue
    }
}
\end{lstlisting}
}
%--------
\subsubsection{数据结构19:CoopeartionManagement\_server}
%--------
\begin{lstlisting}[language=Java, caption=在线协作类]
public static class CoopeartionManagement{
    public static File[] fileBuffer;//协作文档缓冲池
    
    public static int selectFile(File file){
        //查找文件对应的缓冲池的项
    }
    public static void sendFile(File file){
        //向成员发送最新的文档
    }
    public static File receiveContent(){
        //接收修改的内容
    }
    public static File mergeFile(File file){
        //将对文件的增删改合并到当前文档中
    }
    public static void upLoad(File){
        //向数据库上传协作文档
    }
    public static File downLoad(File){
        //从数据库下载协作文档
    }
    
} 
\end{lstlisting}
%--------
\subsubsection{数据结构20:ErrorManagement\_server}
\begin{lstlisting}[language=Java, caption=在线协作类]
public static class ErrorManagement{

    public static Error receiveError(){
        //接收Error
    }
    public static void handleError(){
        //处理能够解决的error
    }
    public static void feedback(Error error){
        //将无法解决的Error反馈给公司
    }

} 
\end{lstlisting}

%--------
{\color{red}

\subsubsection{数据结构21:SurveyManagement\_server}
\begin{lstlisting}[language=Java, caption=信息调研管理类]
public static class SurveyManagement{

    public static Survey LaunchSurvey(){
        //启动调研
    }
    public static void UpdateSurvey(Survey, Question){
        //用户回答问题后,更新调研表的统计信息
    }
    public static Survey FeedbackSurvey(){
        //返回调研结果
    }

} 
\end{lstlisting}


%--------
\subsubsection{数据结构22:ApprovalManagement\_server}
\begin{lstlisting}[language=Java, caption=审批管理类]
public static class ApprovalManagement{
    
    ArrayList<Approval> [] IncompletedApproval;//待完成的审批列表
    ArrayList<Approval> [] completedApproval;//已完成的审批列表

    public static void ApplyFor(Approval approval){
        //申请项目审批
    }
    public static Approval ProgressInquiry(){
        //查询审批进度
    }
    public static Approval GetResult(int ApprovalID){
        //接收审批返回的结果
    }

} 
\end{lstlisting}

%--------
\subsubsection{数据结构23:ThirdPartyManagement\_server}
\begin{lstlisting}[language=Java, caption=第三方账号管理类]
public static class ThirdPartyManagement{
    User[] Users; //与第三方平台有交互的用户列表
    public static void Bind(Int tpID, String tpPassword){
        //绑定第三方平台账号
    }
    public static String fetch(User user){
        //获取第三方平台上的新内容
    }
    public static String analyse(String s){
        //分析获取得到的第三方内容
    }

} 
\end{lstlisting}



}


    %-----------------------------------------------------------------------------
    \section{物理结构设计}
    %-----------------------------------------------------------------------------
        各数据结构无特殊物理结构要求。
    %-----------------------------------------------------------------------------
  
    \section{{\color{red}数据结构与程序模块的关系}}
    %================================ GUIDE ======================================
    % [此处指的是不同的数据结构分配到哪些模块去实现。可按不同的端拆分此表]
    %================================ GUIDE ======================================
{\color{red}  
        \begin{table}[htbp]
            \centering
            \caption{{\color{red}数据结构与程序代码的关系表}}
            \label{tab:datastructure-module}
            \begin{tabular}{|p{6.3em}|p{2em}|p{2em}|p{2em}|p{2em}|p{2em}|
                            p{2em}|p{2em}|p{2em}|p{2em}|p{2em}|p{2em}|}
                \hline %**********************************************************
            ·   & 用户信息管理模块      & 错误信息处理模块  & 云文件管理模块 
                & 文件管理模块          & 日历管理模块      & 团队管理模块      
                & 在线文档协作平台模块  & 通讯管理模块      &
                爬虫模块
                & 流程审批模块          & 信息调研模块
                \\
                \hline
                User                & Y   & Y     & · 
                                    & .   & .     & Y
                                    & .   & Y     & Y
                                    & Y   & Y     \\
                \hline
                Message             & .   & Y     & Y 
                                    & Y   & .     & .
                                    & .   & Y     & .
                                    & .   & .     \\
                \hline
                GrouopMessage       & .   & Y     & Y 
                                    & Y   & .     & Y
                                    & .   & Y     & .
                                    & .   & .     \\
                \hline
                Group               & .   & Y     & . 
                                    & .   & .     & Y
                                    & .   & Y     & .
                                    & Y   & Y     \\
                \hline
                AddressBook         & Y   & Y     & . 
                                    & .   & .     & Y
                                    & .   & Y     & .
                                    & Y   & Y     \\
                \hline
                Friend              & Y   & Y     & . 
                                    & .   & .     & .
                                    & .   & Y     & .
                                    & Y   & Y     \\
                \hline
                Task                & .   & Y     & . 
                                    & .   & .     & Y
                                    & .   & .     & .
                                    & .   & .     \\
                \hline
                Activity            & .   & Y     & . 
                                    & .   & Y     & .
                                    & .   & .     & .
                                    & .   & .     \\
                \hline
                Error               & Y   & Y     & Y 
                                    & Y   & Y     & Y
                                    & Y   & Y     & Y
                                    & Y   & Y     \\
                \hline
                Survey              & .   & .     & . 
                                    & .   & .     & .
                                    & .   & .     & .
                                    & .   & Y     \\
                \hline
                Approval            & .   & .     & . 
                                    & .   & .     & .
                                    & .   & .     & .
                                    & Y   & .     \\
                \hline
                Issue               & .   & .     & . 
                                    & .   & .     & .
                                    & .   & Y     & .
                                    & .   & .     \\
                \hline
                SystemMessage       & Y   & Y     & Y 
                                    & Y   & Y     & Y
                                    & Y   & Y     & Y
                                    & .   & .     \\
                \hline
                \tabincell{l}{UserInfo \\ Management}
                  & Y   & .     & .
                                    & .   & .     & .
                                    & .   & .     & .
                                    & .   & .     \\
                \hline
                \tabincell{l}{Communication \\ Management}
                
                                    & .   & .     & . 
                                    & .   & .     & .
                                    & .   & Y     & .
                                    & .   & .     \\
                \hline
                \tabincell{l}{Calendar \\ Management}
                       & .   & .     & .
                                    & .   & Y     & .
                                    & .   & .     & .
                                    & .   & .     \\
                \hline
                \tabincell{l}{Group \\ Management}
                       & .   & .     & . 
                                    & .   & .     & Y
                                    & .   & .     & .
                                    & .   & .     \\
                \hline
                \tabincell{l}{Cooperation \\ Management }
                 
                                    & .   & .     & . 
                                    & .   & .     & .
                                    & Y   & .     & .
                                    & .   & .     \\
                \hline
            \end{tabular}
            \note{各项数据结构的实现与各个程序模块的分配关系}
        \end{table}
}
%=================================================================================