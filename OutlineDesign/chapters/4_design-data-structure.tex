\chapter{数据结构设计}
%=================================================================================
    \section{逻辑结构设计}
    %-----------------------------------------------------------------------------
        \subsection{用户管理系统数据结构设计}
        %================================ GUIDE ==================================
        % 讲述本系统内需要什么数据结构。这指的是程序运行过程中维护的数据结构。
        % 只是举个例子,此处应和3.3一致。
        %================================ GUIDE ==================================
    %-----------------------------------------------------------------------------
        \subsection{客户端端数据结构}
    %-----------------------------------------------------------------------------
        \subsection{服务器端数据结构}
    %-----------------------------------------------------------------------------
    \begin{lstlisting}[language=Java, caption=示例代码, label={code:first-code}]
    public class Message{
        User sender,receiver;
        String message;
        Time sendtime;
    }
    \end{lstlisting}

    \section{物理结构设计}
    %-----------------------------------------------------------------------------
        各数据结构无特殊物理结构要求。
    %-----------------------------------------------------------------------------
    \section{数据结构与程序模块的关系}
    %================================ GUIDE ======================================
    % [此处指的是不同的数据结构分配到哪些模块去实现。可按不同的端拆分此表]
    %================================ GUIDE ======================================
        \begin{table}[htbp]
            \centering
            \caption{数据结构与程序代码的关系表} \label{tab:datastructure-module}
            \begin{tabular}{|c|c|c|c|}
                \hline
                · & 模块1 & 模块2 & 模块3 \\
                \hline
                结构1 & · & Y & · \\
                \hline
                结构2 & · & Y & · \\
                \hline
                结构3 & · & Y & · \\
                \hline
                结构4 & Y & · & · \\
                \hline
                结构5 & · & · & Y \\
                \hline
            \end{tabular}
            \note{各项数据结构的实现与各个程序模块的分配关系}
        \end{table}
%=================================================================================