\chapter{数据结构设计}
%=================================================================================
    \section{逻辑结构设计}
    %-----------------------------------------------------------------------------
        \subsection{客户端和服务器端通用数据结构}
        %================================ GUIDE ==================================
        % 讲述本系统内需要什么数据结构。这指的是程序运行过程中维护的数据结构。
        % 只是举个例子,此处应和3.3一致。
        %================================ GUIDE ==================================
    %-----------------------------------------------------------------------------
        \subsection{客户端特有数据结构}

    %-----------------------------------------------------------------------------
        \subsection{服务器端特有数据结构}
    %-----------------------------------------------------------------------------
\subsubsection{数据结构1:User}
该数据结构定义了用户的基本属性。
\begin{lstlisting}[language=Java, caption=User定义]
public class User{
    private String ID;//用户ID,唯一
    private String name;//昵称,不唯一
    private String sex="Male";//性别,默认男性
    private Date birthday;//用户的出生日期,可以为空
    private Time createTime;//账户创建时间
    private String address;//用户地址,可以为空
    private String email;//用户绑定的邮箱,可以为空
    public User(String ID){

    }
    public User(String ID,String name,String sex,
        Date birthday,String address,String email){

    }
    public String getID(){

    }
    public String getName(){

    }
    public void setName(String name){

    }
    public String getSex(){

    }
    public void setSex(String sex){

    }
    public Date getBirthday(){

    }
    public void setBirthday(Date birthday){

    }
    public int getAge(){

    }
    public Time getCreateTime(){

    }
    public void setCreateTime(Time createTime){

    }
    public String getAddress(){

    }
    public void setAddress(String address){

    }
    public String getEmail(){

    }
    public void setEmail(String email){

    }
}
\end{lstlisting}

\subsubsection{数据结构2:Message}
该数据结构定义了一对一聊天时发送的信息,所有属性都是不可修改的。
\begin{lstlisting}[language=Java, caption=Message定义]
public class Message{
    private User sender,receiver;//该信息的发送方和接收方
    private String message;//信息内容
    private Time sendtime;//发送时间
    public Message(User sender,User receiver,String message,
        Time sendtime){

    }
    public User getSender(){

    }
    public User getReceiver(){

    }
    public String getMessage(){
        
    }
    public Time getSendtime(){

    }
}
\end{lstlisting}
    
\subsubsection{数据结构3:GroupMessage}
该数据结构定义了群聊时发送的信息,继承自Message类。同样,所有属性都是不可修改的。
\begin{lstlisting}[language=Java, caption=GroupMessage定义]
public class GroupMessage extends Message{
    private Group group;//所属的群聊
    public GroupMessage(Group group,User sender,User receiver,
        String message,Time sendtime){

    }
    public Group getGroup(){

    }
}
\end{lstlisting}

\subsubsection{数据结构4:Group}
该数据结构定义了群聊的基本属性。
\begin{lstlisting}[language=Java, caption=Group定义]
public class Group{
    private String ID;//群ID,唯一
    private String name;//群名称,不唯一
    private User master;//群主
    private User[] managerList;//管理员列表
    private User[] memberList;//群成员列表
    private Scene scene; //场景
    public Group(String ID,User master){

    }
    public String getID(){

    }
    public String getName(){

    }
    public void setName(String name){

    }
    public User getMaster(){

    }
    public User[] getManagerList(){

    }
    public User[] getMemberList(){

    }
    public void appointManager(User user){
    //任命管理员
    }
    public void dismissManager(User user){
    //解除管理员职务
    }
    public void addMember(User user){
    //增加群成员
    }
    public void deleteMember(User user){
    //删除群成员
    }
    public Scene getScene(){

    }
    public void setScene(Scene scene){
        
    }
}
\end{lstlisting}








    %-----------------------------------------------------------------------------
    \section{物理结构设计}
    %-----------------------------------------------------------------------------
        各数据结构无特殊物理结构要求。
    %-----------------------------------------------------------------------------
    \section{数据结构与程序模块的关系}
    %================================ GUIDE ======================================
    % [此处指的是不同的数据结构分配到哪些模块去实现。可按不同的端拆分此表]
    %================================ GUIDE ======================================
        \begin{table}[htbp]
            \centering
            \caption{数据结构与程序代码的关系表} \label{tab:datastructure-module}
            \begin{tabular}{|c|c|c|c|}
                \hline
                · & 模块1 & 模块2 & 模块3 \\
                \hline
                结构1 & · & Y & · \\
                \hline
                结构2 & · & Y & · \\
                \hline
                结构3 & · & Y & · \\
                \hline
                结构4 & Y & · & · \\
                \hline
                结构5 & · & · & Y \\
                \hline
            \end{tabular}
            \note{各项数据结构的实现与各个程序模块的分配关系}
        \end{table}
%=================================================================================