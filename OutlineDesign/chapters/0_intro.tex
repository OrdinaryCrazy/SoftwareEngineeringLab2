\chapter{引言}

\section{编写目的}
在本项目的前一阶段,也就是需求分析阶段,已经将系统用户对本系统的需求做了详细的阐述,这些用户需求已经在上一阶段中对不同用户所提出的不同功能,实现的各种效果做了调研工作,并在需求规格说明书中得到详尽得叙述及阐明。

本阶段已在系统的需求分析的基础上,对即时聊天工具做概要设计。主要解决了实现该系统需求的程序模块设计问题。包括如何把该系统划分成若干个模块、决定各个模块之间的接口、模块之间传递的信息,以及数据结构、模块结构的设计等。在以下的概要设计报告中将对在本阶段中对系统所做的所有概要设计进行详细的说明,在设计过程中起到了提纲挈领的作用。

在下一阶段的详细设计中,程序设计员可参考此概要设计报告,在概要设计即时聊天工具所做的模块结构设计的基础上,对系统进行详细设计。在以后的软件测试以及软件维护阶段也可参考此说明书,以便于了解在概要设计过程中所完成的各模块设计结构,或在修改时找出在本阶段设计的不足或错误。


\section{项目背景}
随着计算机技术的不断发展,网络与每一个人的联系日益紧密。在工作生活中,网络发挥着不可或缺的作用,通过网络进行即时通讯的需求越来越大。
而且,即时通讯并不仅仅是文字信息的接收与发送,我们还提供了许多其他功能。我们可以与朋友进行音视频通讯,还可以共享日程,也可以进行在线的文档协作。我们相信,这些功能可以大大方便日常的工作交流,提高交流的效率。
基于这样的出发点,本项目提供了一个针对团队工作的即时通讯系统。

\section{术语}
[列出本文档中所用到的专门术语的定义和外文缩写的原词组]
暂无
% \begin{table}[htbp]
% \centering
% \caption{术语表} \label{tab:terminology}
% \begin{tabular}{|c|c|}
%     \hline
%     缩写、术语 & 解释 \\
%     \hline
%     % c & d \\
%     % \hline
% \end{tabular}
% \note{这里是表的注释}
% \end{table}