\chapter{出错处理设计}
\section{数据库出错处理}
多重备份时,应采取何种策略,先利用哪一份备份;系统是否暂停服务等。

\begin{itemize}
    \item 当用户信息数据库出错时,立即启动同步备份的数据库,并将其复制到本地。若依然无法恢复,暂停整个用户系统的服务,向客户端的相关操作返回报错信息。同时派专员进行后台修复和安全性检查。向错误信息处理模块发送错误信息。
    \item  当聊天信息数据库出错时,将数据库回退到上一时间,并根据日志重新向用户发送聊天信息。若无法回退或日志丢失,则调用备份数据库,并向用户返回消息发送失败的信息。对于音视频信息,尝试重新连接。同时派专员进行后台修复和网络检查。向错误信息处理模块发送错误信息。
    \item 当文件或云文件数据库出错时,立即启动同步备份的数据库,并将其复制到本地。若依然无法恢复,向用户返回文件丢失的信息。同时派专员后台排查恶意攻击和隐私泄露的情况。向错误信息处理模块发送错误信息。
    \item 当协作文档数据库出错时,将数据库回退到上一时间,并将缓冲池中的数据写回。若无法回退,则调用其他备份数据库。同时派专员进行后台修复和存储检查。向错误信息处理模块发送错误信息。

\end{itemize}


\section{某模块失效处理}
\subsection{用户信息管理模块失效}
若用户信息管理模块失效,则所有操作均无法正常进行。因此,需要对该服务进行多处热备份。失效时向错误信息管理模块发送错误信息,并立即启用备份服务。后台派专员进行错误检查和恢复。若依然无法继续服务,则整个系统暂停服务。
\subsection{通讯管理模块失效}
若通讯管理模块失效,则聊天,邮件,查询聊天记录等功能无法进行。首先检查是否为网络问题,若是,则切换网络并重启。否则,启用备份服务,发送错误信息,等待后台维护人员处理。若依然失效,则向用户聊天窗口,邮件窗口报错,尽快恢复服务。
\subsection{文件管理模块失效}
若文件管理模块失效,则文件上传,下载,收发,日历功能和部分群组功能会受到影响。失效时首先缓存正在处理的文件,向错误信息管理模块发送错误信息,然后启用备份服务,从而尽快恢复服务。
\subsection{云文件管理模块失效}
若云文件管理模块失效,则部分云文件上传,下载,收发功能会受到影响。失效时依然维持其他服务状态,检查具体失效的功能,暂时禁用失效的部分功能。向错误信息管理模块发送错误信息,后台派专员进行错误检查和恢复。
\subsection{日历管理模块失效}
若日历管理模块失效,则日历的修改,团队任务布置等功能受到影响。失效时暂时禁用日历功能,维持其它功能运行。向错误信息管理模块发送错误信息,后台派专员进行错误检查和恢复。
\subsection{团队管理模块失效}
若团队管理模块失效,则团队信息管理,权限管理,协作文档等功能受到影响。失效时维持最小服务状态(即时通讯),向错误信息管理模块发送错误信息,后台派专员进行错误检查和恢复。
\subsection{在线协作平台模块失效}
若在线协作平台模块失效,则在线协作功能受到影响。失效时,首先缓存正在处理的文档,维持其它功能的运行,禁用团队布置文档和用户编辑文档。向错误信息管理模块发送错误信息,后台派专员进行错误检查和恢复。
\subsection{错误信息管理模块失效}
若错误信息管理模块失效,则各个功能部件传输的错误信息可能无法收到,运维人员发布的错误解决方案可能也无法部署。此时,维持系统正在运行的服务,后台派专员进行错误检查和恢复。
